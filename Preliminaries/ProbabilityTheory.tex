Probability theory  is the field that  studies the random phenomena. A random event is the set of  outcomes from an 
experiment conducted under the same conditions  with a variability in results . Probability theory aims to 
describe  this variability. We denote by $\Omega$ the set of observable outcomes, $\omega$, from a
experiment or phenomenon.  However, not every observable event is measurable, so  for the purpose of probability 
theory, a family of subsets from $\Omega$ with particular properties  a ----$\sigma$-algebra is needed. In 
the following, we formalize these concepts.	


\begin{definition}[$\sigma$-algebra]
	Let $\Omega$ a set and $\calF$ a family of subset of $\Omega$, we call $\calF$ a 
	$\sigma$- algebra if the following properties hold:
	\begin{enumerate}[(i)]
		\item $\emptyset \in \calF$,
		\item if $F\in \calF$ then $F^c\in\calF$ where $F^c = \Omega \setminus F$,
		\item if $\displaystyle \{F_i\}_{i=1}^\infty \in \calF $ then 
			$\displaystyle \bigcup_{i\geq 1} F_i \in \calF$.
	\end{enumerate}
\end{definition}

	Let $\calC$ a collection of subsets of $\Omega$. The $\sigma$-algebra generated by $\calC$
denoted by $\sigma(\calC)$, is the smallest $\sigma$-algebra which contains the collection $\calC$, that is
$\sigma(\calC)\supset \calC$, and if $\calB$ is an other $\sigma$-algebra containing $\calC$, then
$\calB \supset \sigma(\calC)$.
\begin{definition}[The Borel $\sigma$-algebra]
	The $\sigma$-algebra generated by the collection of all open sets $U \subset \Omega$ 
\end{definition}
	
	A probability space is a triple $(\Omega, \calF, \P)$ where
\begin{itemize}
	\item $\Omega$ is the set of all possible outcomes of an experiment.
	\item $\calF$ is a conveniently $\sigma$-algebra of subsets of $\Omega$.
	\item $\P$ is a probability measure; that is a function $\P: \calF \to [0,1]$ such that
	\begin{enumerate}[(i)]
		\item
			$\P(A)\geq 0$ for all $A \in \calF$.
		\item
			$\P$ is $\sigma$-additive, that is: If $\{A_n,  n\geq 1 \}$ is a collection of disjoint events,
			then
				$$
					\P \left(
						\bigcup_{n=1}^{\infty} A_n
					\right)= \sum_{n=1}^{\infty}
						\P(A_n).
				$$
		\item
			$\P(\Omega) = 1$.
	\end{enumerate}
\end{itemize}
\begin{definition}[Random Variable]
	Let $(\Omega, \calF, \P)$ be a probability space and $\calB(\R^d)$ the Borel's $\sigma$-algebra. A  function $X:\Omega \to \R^n$ is said to be a random variable if $X$ is $(\calF,\calB(\R^d))$-measurable, that is
	$
		X^{-1}(\calB(\R^d))\subset \calF
	$.
\end{definition}
	Every random variable $X$ induces a probability measure $\mu_X$ on $\R^d$ by
	$$
		\mu_{X}(B) = \P(X^{-1}(B)), \qquad B\in\calB(\R^d).
	$$ 
%

	Having two different measures $\Q$, $\P$, on a measurable space we can transform one measure into the
other via  Radon-Nikodym theorem  (see for example \cite[Thm. 10.1.2]{Williams1991}).

\begin{thm}[Radon-Nikodym]
	Let $\P$ and $\Q$ probability measures on the measurable space $(\Omega,\calF)$. Suppose that for all
	$B\in\calF$ $\Q(B)=0$ implies $\P(B)=0$. Then there exist a integrable random variable $X$ such that
	$$
		\Q(E) = \int_{E}Xd\P, \qquad \forall E\in\calF.
	$$
	$X$ is $\P$-a.s. unique and is written as $\displaystyle X = \frac{d\Q}{d\P}$.
\end{thm}
	This important result describes the density $p$ of a random variable $X$ as the $\P$-a.s.
unique Radon-Nikodym derivative of the induced distribution $\mu_{X}$ w.r.t. Lebesgue measure, in other 
words
$$
	\mu_{X}(B)=\int_{B} p(x)dx.
$$

\begin{definition}[Expectation]
	Let $X$ be a integrable random variable on a probability space $(\Omega, \calF, \P)$. Then the expectation
	of $X$ is defined by
	\begin{equation*}
		\EX{X}= \int_{\Omega} X(\omega) \P(d\omega).
	\end{equation*}
\end{definition}
