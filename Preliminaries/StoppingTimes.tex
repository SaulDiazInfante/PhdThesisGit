%
Essentially, a stopping time provides a way to verify the first occurrence of an random event. This will be useful 
to justify the results presented on \Cref{ch:Chapter4}. We enunciate the formal definition and
two results to assure its random meaning. 

\begin{definition}[Stopping Time]
	A random variable $\tau:\Omega \to [0, \infty]$ is called an $\{\calF_{t}\}$-stopping time if
	$\{\omega: \tau(\omega)\leq t \}\in \calF_t$ for any $t\geq 0$.
\end{definition}

\begin{thm}
	If $\{ X_t\}_{t\geq 0}$ is a progressively measurable process and $\tau$ is a stopping time,
	then $X_t \1{\tau <\infty}$ is $\{\calF_t \}$-measurable.
\end{thm}
%
\begin{thm}
	Let $\{X_t \}_{t\geq 0}$ be and $\R^d$-valued continuous $\{\calF_t\}$-adapted process and $D\subset \R^d$ an open 
	set. Then
	$
		\tau := \inf\left\{t \geq 0: X_t \notin D \right\}
	$
	is an $\{\calF_t\}$-stopping time.
\end{thm}
%
%\begin{lem}[Fatou]
%	For any non negative measurable functions $\{ X_k\}_{k\geq 1}$ on $(\Omega, \calF, \P)$, we have
%	$$
%	\EX{
%		\liminf_{k\to\infty}
%		X_k
%	} \leq \liminf_{k\to\infty}\EX{X_k}.
%	$$
%\end{lem}
