
	Conditional expectation plays a very important role in the modern probability theory. It gives sense and 
foundation for the definitions of martingales and Markov processes. In fact, other areas of probability 
as stochastic dynamics, conditioning permits to describe and to analyze dynamical systems with randomness.
Roughly speaking, the conditional expectation is an average that considers only a portion of information.
Assume $X\in L_1(\Omega,\calF,\P)$ and let $\calG \subset \calF$ a sub-$\sigma$-algebra. Then the
conditional expectation of the random variable $X$ given $\calG$ is the new random variable 
$Y = \cEX{X}{\calG}$ such that
\begin{enumerate}[(i)]
	\item
		$\cEX{X}{G}$ is $\calG$-measurable and integrable.
	\item
		For all event $G\in \calG$ we have 
		$
			\displaystyle
			\int_{G} X d\P = \int_{G} \cEX{X}{\calG}d\P.
		$
\end{enumerate}
	This new random variable is unique in the sense that if there is an other $\tilde{Y}$ satisfying
the same two above properties, then $\P{[Y\neq \tilde{Y}]}=0$. In this case $\tilde{Y}$ is said to be 
a version of the conditional expectation $\cEX{X}{\calG}$.
Now we list some standard properties of the conditional expectation \cite{Williams1991}.
Here $X_1$, $X_2$, $Z$ are integrable random variables, $a_1$,$a_2\in \R$, and $\calG$, $\calH$ 
are sub-$\sigma$-algebras of $\calF$.
\begin{enumerate}[(E1)]
	\item
		If $Y$ is any version of $\cEX{X}{\calG}$, then $E[X]=E[Y]$.
	\item
		If $X$ is $\calG$ measurable, then $\cEX{X}{\calG} = X$, $\P$-a.s.
	\item
		(Linearity) $\cEX{a_1X_1+a_2X_2}{\calG} = a_1\cEX{X_1}{\calG} + a_2\cEX{X_2}{\calG}$ $\P$-a.s.\\
		Clarification:
			if $Y_1$ is a version of $\cEX{X_1}{\calG}$ and $Y_2$ is a version of $\cEX{X_2}{\calG}$, then
			$a_1Y_1 + a_2Y2$ is a version of $\cEX{a_1 X_1 + a_2X_2}{\calG}$.
	\item
		(Positivity) If $X \geq 0$, then $\cEX{X}{\calG} \geq 0$,  $\P$-a.s.
	\item
		(Conditional Monotone Convergence)
		If $0 \leq X_n \uparrow X$, then $\cEX{X_n}{\calG} \uparrow \cEX{X}{\calG}$, $\P$-a.s.
	\item
		(Conditional Fatou)
		If $X_n\geq 0$, then 
		$\displaystyle
			\cEX{\liminf_{n \to \infty} X_n}{\calG}
			\leq
			\liminf_{n \to \infty}
			\cEX{X_n}{\calG}
		$.
	\item
		(Conditional Dominated Convergence)
		If $|X_n(\omega)|\leq V(\omega)$ for all $n$, $\EX{V}<\infty$, and $X_n \to X$ $\P$-a.s., then
		$\cEX{X_n}{\calG} \to \cEX{X}{\calG}$, $\P$-a.s.
	\item
		(Conditional Jensen)
		If $f$ is a real-valued convex function, then
		$$
			f
			\left(
				\cEX{X}{\calG}
			\right)
			\leq
			\cEX{f(X)}{\calG}.
		$$
		\item
			(Tower Property) If $\calH$ is a sub-$\sigma$-algebra of $\calG$, then
			$$
				\cEX{\cEX{X}{\calG}}{\calH} = \cEX{X}{\calH}, \qquad \P\text{-a.s.}.
			$$
		\item
			If $Z$ is $\calG$-measurable and bounded, then
			$
				\cEX{ZX}{\calG} = Z\cEX{X}{\calG}.
			$
		\item
			If $X$ is independent from $\calH$, then
			$\cEX{X}{\calH} = \EX{X}$, \qquad $\P$-a.s.
\end{enumerate}

	Now consider a filtered space $\left(\Omega, \calF, \{\calF_t\}_{t\geq 0}, \P \right)$ and define
$$
	\calF_{\infty}:= \sigma
		\left(
			\bigcup_{t\geq 0}
			\calF_t
		\right) \subset \calF.
$$

\begin{definition}[Martingale]
	A process $\{M_t\}_{t\geq 0}$ is called a martingale (relative to $\left(\{F_t \}_{t\geq 0}, \P \right)$ if
	\begin{enumerate}[(i)]
		\item
			$M$ is adapted,
		\item
			$\EX{|M_t|}<\infty$ \qquad for all $t\geq 0$,
		\item\label{dfn:Martingale}
			$
				\cEX{M_t}{\calF_s} = M_s
			$, \qquad $\P$-a.s., \qquad $0\leq s \leq t$.
	\end{enumerate}
\end{definition}

	In this way, a \emph{supermartingale} (relative to $\left(\{F_t \}_{t\geq 0}, \P \right)$ is defined similarly, 
except that \eqref{dfn:Martingale} is replaced by
$$
	\cEX{M_t}{\calF_s}\leq M_s
	\qquad \P\text{-a.s.},
	\qquad 0\leq s \leq t, 
$$ and a \emph{submartingale} is defined with \eqref{dfn:Martingale} 
replaced by
$$
	\cEX{M_t}{\calF_s} \leq M_s
	\qquad \P\text{-a.s.},
	\qquad 0\leq s \leq t.
$$ 
\begin{thm}
	Let $\{M_t\}_{t\geq 0}$ be and $\R^d$-valued martingale with respect to $\{\calF_t\}$, and let $\theta$, $\rho$ two 
	finite stopping times. Then 
	$$
		\EX{M_{\theta}}{\calF_{\rho}}= M_{\theta \wedge\rho}.
	$$
\end{thm}
\begin{definition}[Local Martingale]
	An $\R^d$-valued $\{F_t\}$-adapted integrable process $\{M_t\}_{t\geq 0}$ is said to be a \emph{local martingale}
	if there exists a nondecreasing sequence $\{\tau_k\}_{k\geq 1}$ of stopping times with $\tau_k \uparrow \infty$
	$\P$-a.s. such that $\{M_{\tau\wedge t} - M_0 \}$ is a martingale.
\end{definition}
A fundamental process in this thesis is the Brownian Motion.
\begin{definition}[Brownian Motion]
	Let $(\Omega, \calF, \P)$ be a probability space with filtration $\{\calF_t\}_{t\geq 0}$. A standard unidimensional
	Brownian motion is a real-valued continuous adapted process $\{W_t\}_{t\geq 0}$ which satisfies:
	\begin{enumerate}[(i)]
		\item 
			$W_0=0$, \qquad $\P-\as$;
		\item 
			the increments $W_t-W_s$ are normally distributed with mean zero and variance $t-s$ for 
			$0\leq s\leq t<\infty$;
		\item
			$W_t-W_s$ is independent of $\calF_s$.
	\end{enumerate}
\end{definition}

	Consider a Brownian motion $\{W_t\}_{t\geq 0}$  and a sequence of times $0 \leq t_0 < t_1 < \ldots < t_k < \infty$.
Results that $\{W_t\}_{t\geq 0}$ has independent increments, that is, the random variables $W_{t_i}-W_{t_{i-1}}$ 
$1\leq i \leq k $ are independent. Moreover, the distribution of $W_{t_i} - W{t_{i-1}}$ depends only on the 
difference $t_i - t_{i-1}$, in this sense, we say that the Brownian motion has stationary distribution. With this in
mind, also we can say that this process is a martingale. As we will see, the above is fundamental for the numerical 
approximations of SDEs.
%
