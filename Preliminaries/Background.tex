
	A stochastic process $X$ is a system which could stay at each moment on any state of a given set $S$
\begin{definition}[Stochastic Process]
	A stochastic process is a collection of random variables $X=\{X_t: t \in T\}$ on $(\Omega,
	\calF)$,  which takes values in a measurable space $(S,\calS)$, and where the index $t\in [0,\infty)$, 
	conveniently receive an interpretation as time. Thus for a fixed $\omega \in \Omega$, the function
	$X_t(\omega)$, $t\geq 0$ is a sample path of the process $X$ associated with $\omega$,  and for any
	fixed $t$ , $X_t(\omega)$, $\omega \in \Omega$ is a random variable. 

\end{definition}
 The main purpose of this thesis deals with the numerical approximation of sample paths.
\begin{definition}[Measurable Process]
	A stochastic process $X$ is measurable if the mapping
	\begin{equation*}
		(t, \omega) \to X_t(\omega):
		\left(
			[0,\infty)\times \Omega,
			\calB \left(
				[0,\infty)
			\right) \otimes \calF
		\right)
		\to 
		\left(
			\R^d, \calB \left (\R^d \right)
		\right)
	\end{equation*}
	is measurable.
\end{definition}
%
	We equip the underlying sample space $(\Omega, \calF)$ with a filtration $\{\calF\}_{t\geq 0}$ in order to 
keep track information about the past, present and future of a stochastic process. Formally, a filtration is
a nondecreasing family of sub-$\sigma$-algebras of $\calF$ such that 
$\calF_s\subseteq \calF_t \subseteq \calF$ for $0\leq s \leq t <\infty$ and is called right continuous if
$ %\displaystyle
	\calF_t = \bigcap_{r>t} \calF_r
$
for all $t\geq 0$. Thus if the underlying probability space is complete,
right continuous and $\calF_0$ contains all $\P$-null sets, then we say that the filtration
$\{ \calF_t\}_{t\geq 0}$ satisfies the usual conditions. In the following, we will work only on a
complete probability space $(\Omega, \calF, \P)$  with a filtration $\{\calF_t \}_{t \geq 0}$ which verifies
the usual conditions.
	
	Given a stochastic process, the simplest choice of a filtration is that generated
by the process itself, i.e. 
$
	\calF^{X}_t:=\sigma(X_s; 0\leq s\leq t)
$
the smallest $\sigma$-algebra with respect to which $X_s$ is measurable for every $s\in[0, t]$.
The introduction of this concept gives sense to the following.
\begin{definition}[Adapted Process]
		We call a process \it{adapted} to the filtration $\{\calF\}_{t\geq 0}$  if, for each $t>0$ fixed $X_t$  is a
	$\calF_t$-measurable random variable.
\end{definition}
Clearly, every process $X$ is adapted to $\{ \calF_t^{X}\}$.
\begin{definition}[Progressively Measurable Process]
	The stochastic process $X$ is progressively measurable if the mapping
	$$
		(s,\omega)\to X_s(\omega):
		\left(
			[0,t] \times \Omega,
			\calB([0,t]) \otimes \calF_t
		\right) \to 
		\left(
			\R, \calB\left(\R^d\right)
		\right)
	$$
	is measurable for each $t\geq 0$, that is, if, for each $t>0$ and $A\in \calB\left(\R^d\right)$, 
	the set 
	$$
		\{
			(\omega, s) : 0 \leq s \leq t, \omega \in \Omega, X_s(\omega) \in A
		\}
	$$
	belongs to the product $\sigma$-algebra
	$
		\calB([0,t]) \otimes \calF_t
	$.
\end{definition}
%