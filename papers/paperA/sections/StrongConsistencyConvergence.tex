
		It is always important in the construction of  new algorithms to study  the global
	discretization error and give an estimation of the speed of convergence. Here, they are
	carried out with  the analysis of the properties of consistency and convergence, see
	\cite{Kloeden1992}. For simplicity, we study these properties for a
	one-dimensional autonomous SDE
	\begin{equation}\label{eqn:autonomousSDE}
		dy(t) = f(y(t))dt+g(y(t))dW(t),
	\end{equation}
	satisfying the necessary conditions of existence and uniqueness of solution. 

	So, considering \Cref{dfn:Consistency} and \Cref{thm:ConsistencyConvergence} we prove 
	convergence of the explicit Steklov approximation via strong consistency. 
	\begin{thm}\label{thm:Consistency}
		A time discrete approximation of SDE 
	\eqref{eqn:autonomousSDE} generated with the explicit Steklov method 
	\eqref{Steklov} is strongly convergent.
	\end{thm}
	\begin{proof}
		We substitute the Steklov recurrence \eqref{Steklov} in the left hand side of the
	inequality \eqref{eqn:DefConsitenceB}. Given that $F$, $G$ and $\Psi_h$ are continuous
	functions adapted to the filtration $(\mathcal{F}_t)_{t\in[0,T]}$ and using standard
	conditional expectation properties \cite{Williams1991}, it follows that:
	\begin{align*}
		\mathbb{E} \left(	\left| \mathbb{E} \left( \frac{Y_{n+1}-Y_n}{h}
		\left|\mathcal{F}_{t_n}\right.\right)-F \left(Y_n\right)\right|^2\right)
		&= \mathbb{E} \left(\left|\frac{\Psi_h(Y_n)-Y_n}{h}-F(Y_n)\right|^2
		\right)\\
		&= \mathbb{E} \left(\left|\frac{H^{-1}(H(Y_n)+h)-H^{-1}(H(Y_n))}{h}
		-F(Y_n)\right|^2\right).
	\end{align*}
	Since the functions $F$ and $\Psi_h$ are Lipschitz we can apply the {\it Inverse
	Function} theorem and from  \eqref{H}, we have that
	$$
		(H^{-1})'(H(Y_n))=F(Y_n),
	$$
	then given any $\epsilon>0$  there exists $\delta=\delta(\epsilon)$ such that whenever
	$0<h<\delta(\epsilon)$ then
	$$
		\left|
		\frac{H^{-1}(H(Y_n)+h)-H^{-1}(H(Y_n))}{h}
		-F(Y_n)
		\right|<\epsilon.
	$$
	So, taking $\epsilon=\sqrt{h}$ and $c(h)=h(\delta(\sqrt{h}))^2$, the  condition (ii)
	is satisfied. With an analogous procedure, the condition (iii) is verified and the
	condition (i) follows straightforward from the definition of $c(h)$.
	\end{proof}
	
		Thus,  we can ensure that the explicit Steklov scheme converges on bounded time
	intervals \cite{Robinson2002}. However, if we are interested in simulating the
	solution of the SDE \eqref{eq1} for large periods of time, we need to use stable
	methods. We can interpret the stability of a numerical method, in some sense, as its
	capacity to preserve the  dynamical structure of the solution in that sense.  In the
	next two sections, we study the stability of the explicit Steklov method
	\eqref{Steklov} in mean and mean square sense and extend this study in a path-wise
	sense for the additive case.