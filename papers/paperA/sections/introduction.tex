	In this chapter, we focus on the following scalar stochastic differential equation
\begin{equation}\label{eq1}
	dy(t)=f(t,y(t))dt+g(t,y(t))dW(t), \qquad y_0=y(0),
\end{equation}
considering the drift term as $f(t,y(t))=f_1(t)f_2(y(t))$. Given this functional 
form of $f$, we propose an exact
explicit algorithm for solving the  deterministic equation linked to (\ref{eq1});
details of this exact differentiation are given in \cite{Matus2005}. So, the main
characteristic of this new method is that it preserves qualitative features of the
deterministic solution associated  to the SDE. Next, we prove strong consistency,
convergence and study the linear stability of the proposed method using properties
of the {\it Steklov mean} \cite{Timan1963}.  Moreover,  we analyze the nonlinear
stability of the Steklov stochastic approximation  specifically the asymptotic
mean-square stability in the multiplicative case and the  path-wise stability in the
additive case. Finally, we show  the efficiency of the new  scheme in numerical
problems with harsh requirements of  stability like the logistic  equation for the
multiplicative case and the Langevin  equation with a particular potential  for the
additive case.

	In \cref{sec:SteklovMethod}, we construct the explicit
Steklov method  for the SDE \eqref{eq1} and show its development  with some examples.
In the next section, we prove strong consistency and convergence of the new explicit
method. In \cref{sec:LinearStability}, sufficient conditions for the asymptotic mean and
mean-square  stability are given for both additive and multiplicative cases. A
nonlinear  stability  analysis is carried out in \cref{sec:NonlinearStability}, where we prove
that the  explicit Steklov  approximation is asymptoticly stable in square mean sense
in the  multiplicative case and  it is path-wise stable under certain conditions in
the additive  case. In \cref{sec:NumericalResults},  we test the Steklov method for
the stochastic  logistic equation in the  multiplicative case and and for the Langevin
equation in  Brownian dynamics. Also, we show  numerical evidence that the Steklov
method  is successful  with step sizes significantly  large reaching larger time
scales of simulation. Finally,  we give some conclusions.