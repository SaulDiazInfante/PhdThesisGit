%
	For simplicity, we begin the construction of the Linear Steklov (\SM) method  considering the scalar case of SDE 
\eqref{eqn:SDE1}, that is, when $d=m=1$, also, to shorten notation we use $a,b$ instead $a_j,b_j$. 
Let $0=t_0 < t_1< \cdots < t_N=T$ a partition of the interval $[0,T]$ with constant step-size $h=T/N$ and such that
$t_k=kh$ for $k=0,\ldots, N$. The main idea of the  \SM approximation  consists in 
estimating the drift coefficient of \eqref{eqn:SDE1}  by
\begin{equation}\label{eqn:LinerizedSteklovMean}
	f(y(t)) \approx 
		\varphi_{f}(y(t_{\eta_{+}(t)})) =
		\left(
			\frac{1}{y(t_{\eta_+(t)})-y(t_{\eta (t)})}
			\bigint \limits
_				{y(t_{\eta(t)})}^{y(t_{\eta_+(t)})}
					\frac{du}
						{
							a(y(t_{\eta(t)}))u
							+b
						}
	\right)^{-1}, \qquad t\in [0,T],
\end{equation}
where
\begin{align*}
	\eta(t) &:=
	k\text{  for } t\in [t_k, t_{k+1}), \quad k\geq 0,\\
	\eta_{+}(t) &:= 
	k+1  \text{ for } t\in [t_k, t_{k+1}), \quad k\geq 0.
\end{align*}
So we define the \SM method for the scalar version of the SDE \eqref{eqn:SDE1} using a split-step formulation as follows
\begin{align}
	Y_k^{\star} &= Y_k + h \varphi_f(Y^{\star}_k), \label{eqn:SSLSM1}\\
	Y_{k+1}	&= Y_k^{\star} + g(Y_k^{\star})\Delta W_k \label{eqn:SSLSM2},
\end{align}
with $Y_0=y_0$ and  $\varphi_{f}\left(Y_k^{\star}\right)$ defined by 
\begin{equation}
	\varphi_{f}\left(Y_k^{\star}\right)
	=
		\left(
			\frac{1}{Y_{k}^{\star}-Y_{k}}
			\int 
			%\limits
			_{Y_{k}}^{Y_{k}^{\star}}
				\frac{du}
				{
					a(Y_k)u
					+b
				}
	\right)^{-1},
\end{equation}
where $\varphi_{f}$ is the linearized Steklov average \cite{Diaz-Infante2015,Matus2005}. For higher dimensions, we 
adapt the same split step scheme \crefrange{eqn:SSLSM1}{eqn:SSLSM2} as follows. 
For each component equation $j\in\{1, \ldots,  d\}$, on the iteration $k\in\{1, \ldots,  N\}$ take
\begin{align}
	a_{j,k} &=
	a_j
	\left(
		Y^{(1)}_{k},
		\ldots, Y^{(d)}_{k}
	\right),
	&
	b_{j,k} =
	b_{j}
	\left(
			Y^{(-j)}_k
	\right).
\end{align}
So, define  
$
	\varphi_{f}(Y_k^{\star})=
		\left(
			\varphi_{f^{(1)}}(Y_k^{\star}),
			\ldots,
			\varphi_{f^{(d)}}(Y_k^{\star})
		\right)
$
by
\begin{equation}
	\varphi_{f^{(j)}}\left(Y_k^{\star}\right)
		=
		\left(
			\frac{1}{Y_{k}^{\star(j)}-Y_{k}^{(j)}}
			\int 
			%\limits
				_{Y_{k}^{(j)}}^{Y_{k}^{\star(j)}}
				\frac{du}
				{
					a_{j,k} u
					+b_{j,k}
				}
		\right)^{-1}.
\end{equation}
%
	It is worth mentioning that even this formulation is semi implicit, we always can derive a explicit version. 
The next result deals with this issue. To simplify notation, we define  $A^{(1)}= A^{(1)}(h,u)$,  $A^{(2)}=A^{(2)}(h,u)$  and $b=b(u)$ by 
\begin{align}\label{eqn:SolutionFunctions}	
	A^{(1)}&:=
		\begin{pmatrix}
			e^{ha_1(u)} & \multicolumn{2}{c}{\text{\kern0.5em\smash{\raisebox{-1ex}{\huge 0}}}} \\
			&\ddots\\
			\multicolumn{2}{c}{\text{\kern-0.5em\smash{\raisebox{0.95ex}{\huge 0}}}} 
			& e^{ha_d(u)}
		\end{pmatrix},
		\notag
		\\
		%	
	A^{(2)}&:=
	\begin{pmatrix}
		\left(
			\displaystyle
			\frac{e^{ha_1(u)} - 1}{a_1(u)}
		\right)\1{E_1^c}	& 
		\multicolumn{2}{c}{\text{\kern0.5em\smash{\raisebox{-1ex}{\huge 0}}}}\\
		 & \ddots&\\
		\multicolumn{2}{c}{\text{\kern0.5em\smash{\raisebox{-1ex}{\huge 0}}}}&
		\left(
			\displaystyle
			\frac{e^{ha_d(u)} - 1}{a_d(u)}
		\right)\1{E_d^c}% + h \1{E_i} 
	\end{pmatrix}
	+h
	\begin{pmatrix}
		\1{E_1} & \multicolumn{2}{c}{\text{\kern0.5em\smash{\raisebox{-1ex}{\huge 0}}}}\\
		&\ddots &\\
		\multicolumn{2}{c}{\text{\kern0.5em\smash{\raisebox{-1ex}{\huge 0}}}} &
		\1{E_d}
	\end{pmatrix},\\	
	E_j&:=\{x \in \R^d: a_j(x)=0\} , \qquad 
	b(u):= \left(
		b_1(u^{(-1)}), \dots , b_d(u^{(-d)})
	\right)^T.		
	\notag
\end{align}
	Also we will need the following results from \cite[Thm 2.1]{Lawlor2012}, \cite[Thm. 1]{FineAIandKass1966}.
The first theorem will help us  with the singularities of set $E_j$ in the case where all elements of this set
are limit points. 
\begin{thm}[Multivariate L'h\^{o}pital's Rule] \label{thm:Lawlor}
	Let $\mathcal{N}$ be a neighborhood in $\R^2$ containing a point $\mathbf{q}$ at which
	two differentiable functions $f:\mathcal{N}\to \R$ and $g:\mathcal{N}\to \R$ are zero.
	Set 
	$$
		C=\{x \in \mathcal{N}: f(x)=g(x)=0 \},
	$$
	and suppose that $C$ is a smooth curve through $\mathbf{q}$.
	Suppose	there exist a vector $\mathbf{v}$ not tangent to $C$ at $\mathbf{q}$
	such that the directional derivative $D_{\mathbf{v}}g$ of $g$ in the direction of $\mathbf{v}$ is never zero
	within $\mathcal{N}$. Also we assume that $\mathbf{q}$ is a limit point of $\mathcal{N}\setminus C$. Then
	\begin{equation*}
		\lim_{(x,y)\to \mathbf{q}}
		\frac{f(x,y)}{g(x,y)} =
		\lim_{
				\substack{
					(x,y)\to \mathbf{q}\\ 
					(x,y)\in \mathcal{N} \setminus C
				}
		}
		\frac{D_{\mathbf{v}} f }{D_{\mathbf{v}} g},
	\end{equation*}
	if the latter limit exists.
\end{thm}
For the second theorem we will need the following concepts.
%
With this on mind, we additionally require the following. 
\begin{hypothesis}\label[hypothesis]{ass:ajBound} %\label{as:aiFunctions}
	For each component function $f^{(j)}:\R^d:\to \R$ %$j \in \{1, \dots, d \}$,
	with $j \in \{1,\dots, d\}$:
	\begin{enumerate}[({A}-1)]
		\item\label{ass:FunctionStructure}
		There are two locally Lipschitz functions of 
		class $C^1(\R^d)$ denoted by
		$a_j:\mathbb{R}^{d} \to \mathbb{R}$ and
		$b_{j}:\mathbb{R}^{d-1} \to \mathbb{R}$ such that 
		the $j$-component of the drift function can be rewritten 
		as in \eqref{eqn:AlternativeConstruction}.
		\item\label{a}
		There is a positive constant $L_a$ such that
		$$
		a_{j}(x) \leq L_a, \qquad\forall x\in \R^d.
		$$
		\item Each function $b_j(\cdot)$ satisfies the linear growth condition
		\begin{equation*}
			|b_j(x^{(-j)})|^2 \leq L_{b}(1+|x|^2) , \qquad \forall x\in \R^d.
		\end{equation*}
	\end{enumerate}
\end{hypothesis}

\begin{hypothesis} \label{ass:HypThmSingularities}
	The set $E_j:=\{x\in \R^{d}: a_j(x)=0\}$ satisfies either:
	\begin{enumerate}[(i)]
		\item
			All point $q \in E_j$ is a non isolated zero of $a_j$ and:
			\begin{itemize}
				\item the set 
					$$
						D:=\{u \in B_r(q): e^{ha_j(u)}-1=a_j(u)= 0\},
					$$ 
					is a smooth curve through $q$. 
				\item
					The canonical vector $e_j$ is not
					tangent to $D$.
				\item
					For each $q \in E_j$, there is an open ball with center
					on $q$ and radio $r$ $B_r(q)$, such that  
					and
					$$
						a_j\neq 0, \qquad
						\frac{\partial a_j(u)}{\partial u^{(j)}} \neq 0 ,\qquad 
						\forall u \in B_r(q)
						\setminus D.
					$$
			\end{itemize}	
		\item
			All point $q \in E_j$ is a isolated zero of $a_j$ and:
			\begin{itemize}
				\item
					For each $q\in E_j$,  $q$ is not a limit point of the set 
					$E_{\alpha}:=\{x \in \R^d: (a_j)_\alpha(x)=0\}$.
				\item
					For each $q \in E_j$ there is a star-like set respect to $q$ $E_q$, such that
					the directional derivative respect to $q$ satisfies
					$$
						 (a_j)_\alpha(x) \neq 0, \qquad \forall x\in E_q.
					$$
			\end{itemize}		
	\end{enumerate}	
\end{hypothesis}
%
By \Cref{ass:ajBound} there is a unique linear Steklov approximation and by \Cref{ass:HypThmSingularities}
we can apply  \Cref{thm:Lawlor} or \Cref{thm:Fine} to deals with possible singularities of the matrix function 
$A^{(2)}$ defined on \eqref{eqn:SolutionFunctions}. Under the previous  assumptions 
we will show that the explicit Linear  Steklov approximation \crefrange{eqn:SSLSM1}{eqn:SSLSM2}
exists, the function $\varphi_f$ is bounded by the drift function $f$ and also the 
coefficients $\varphi_f$  and $g$  satisfy a monotone condition. First, we will give the following lemma.
\begin{lem}\label{l1}
	Assume \Cref{ass:OSLC,ass:ajBound,ass:HypThmSingularities} hold. 
	The function $\Phi_j(x)=\Phi(h, a_j)(x)$ 
	defined by
	\begin{equation}\label{eqn:ExpBound}
		\Phi_j(x):=\frac{e^{ha_j(x)}-1}{ha_j(x)},
	\end{equation}
	is bounded  on $\R^d$ for each $j\in \{ 1,\dots, d\}$
	by a positive constant $L_{\Phi}$, which could depend on $h$.
\end{lem}
\begin{proof}    By \Cref{ass:OSLC}, the operator $\Phi$ is continuous
	on $E_j^c$, thus 
	\begin{equation}\label{e0}
	\lim_{h\to 0}
	\frac{e^{ha_j(x)}-1}{ha_j}=1,
	\end{equation}
	for each fixed $\in E_j^c$. If $x^*\in E_j$ and fixing any $h$, by \Cref{ass:HypThmSingularities}, 
	we obtain one of the following cases:
	\begin{equation}\label{eqn:ajSingularityCasei}
	\lim_{
		\substack{
			x \to x^*\\ 
			x\in E_j^c
		}
	}
	\Phi(h,a_j)(x) =
	%
	\lim_{
		\substack{
			x \to x^*\\ 
			x\in E_j^c
		}
	}	
	\frac{\frac{\partial a_j(x)}{\partial x^{(j)}} 
		h e^{h a_j(x)} 
	}{
	h\frac{\partial a_j(x)}{\partial x^{(j)}}
}=1,
\end{equation}	
or
\begin{equation}\label{eqn:ajSingularityCaseii}
\lim_{
	\substack{
		x \to x^*\\ 
		x\in E_j^c
	}
}
\Phi(h,a_j)(x) 
=
%
\lim_{
	\substack{
		x \to x^*\\ 
		x\in E_j^c
	}
}	
\frac{
	\left(
	e^{h a_j(x)} - 1
	\right)_{\alpha}
}{
\left(
h a_j(x)
\right)_{\alpha}
}	=	1, \qquad \alpha = 0,\pi, 2\pi,\dots
\end{equation}	
From \eqref{e0}, \eqref{eqn:ajSingularityCasei} and \eqref{eqn:ajSingularityCaseii} we can deduce that 	
\begin{equation}\label{eqn:PhiBound}
	\left|\frac{e^{ha_j(x)}-1}{ha_j(x)}
	\right|\leq\left|\frac{e^{hL_a}-1}{ha^*_j}
	\right|,
	\qquad \forall x \in \R^d,
\end{equation}
where $a^*_j:= \inf_{x\in E_j^c}\{|a_j(x)|\}$. So, for each $h$ fixed by inequality \eqref{eqn:PhiBound} we can 
deduce that there a positive constant $L_{\Phi}=L_{\Phi}(h)$ such that
$$
	|\Phi_j(x)|\leq L_{\Phi}, \qquad \forall j\in \{1,\dots, d\}.	
$$ 
Finally, if $a^*_j=0$, then we can
use an argument similar to \crefrange{eqn:ajSingularityCasei}{eqn:ajSingularityCaseii}.	
\end{proof}

Now we can state the following result.
\begin{lem}\label{lem:PhiFhProp}
	Let \Cref{ass:OSLC,ass:ajBound,ass:HypThmSingularities} holds, and $A^{(1)}$, $A^{(2)}$, $b$  
	defined by 
	\eqref{eqn:SolutionFunctions}. Then given $u\in\mathbb{R}^d$, the equation
	\begin{equation}\label{eqn:varphiEquation}
		v = u + h \varphi_f(v),
	\end{equation}
	has a unique solution 
	\begin{equation}\label{eqn:varphiEqnSolution}
		v = A^{(1)}(h,u)u +A^{(2)}(h,u) b(u)	.
	\end{equation}
%
	If we define the functions
	$F_h(\cdot)$, $\varphi_{f_h}(\cdot)$ and $g_h(\cdot)$ by
	\begin{equation}\label{eqn:FunctionshDefinition}
		F_h(u) = v,
%			\left(
%				u + \frac{b}{a(u)}
%			\right)\exp\left(ha(u)\right)
%				-\frac{b}{a(u)},
			\qquad 
			\varphi_{f_h}(u) =\varphi_{f}(F_h(u)),
			\qquad
			g_h(u) = g(F_h(u)),
	\end{equation}
	then $F_h(\cdot)$, $\varphi_{f_h}(\cdot)$, $g_h(\cdot)$ are local Lipschitz functions 
	and for all $u\in \mathbb{R}^d$ and each $h$ fixed, there is a positive constant $L_{\Phi}$ such that
	\begin{equation}\label{eqn:PhifhFbound}
		|\varphi_{f_h}(u)|\leq L_{\Phi} |f(u)|. 
	\end{equation} 
	Moreover, for each $h$ fixed,
	%$g_h$ is a locally Lipschitz function and 
	 there are positive constants $\alpha^*$ and  $\beta^*$ such that
	\begin{equation}\label{eqn:h-MonotoneCondition}
		\innerprod{\varphi_{f_h}(u)}{u} \vee |g_h(u)|^2 \leq \alpha^* + \beta^* |u|^2, 
		\qquad
		\forall u \in \R^d.
	\end{equation}
\end{lem}
%
\begin{proof}
	Let us first prove that \eqref{eqn:varphiEqnSolution} is solution 
	of equation \eqref{eqn:varphiEquation}.  Note that 
	\begin{equation}\label{e1}
	v^{(j)} = u^{(j)} + h \,\varphi_{f^{(j)}}(v),	
	\end{equation}
	for each $j\in \{1,\dots, d\}$ and using the linear Steklov function \eqref{eqn:LinerizedSteklovMean}, 
	we can derive that
	\begin{equation}\label{r1}
		v^{(j)}= e^{h a_j(u)} u^{(j)} + 
		\left[
			h\Phi_j(u)
			\1{E_j^c}
			+h \1{E_j}
		\right]b_{j}(u^{(-j)}),
	\end{equation}	
which is the $j$-component of the vector
$A^{(1)}u +A^{(2)} b(u)$. Now let us prove inequality \eqref{eqn:PhifhFbound}. Given that 
$v =\varphi_{f}(F_h(u))$, 
we can also rewrite \eqref{e1} as
$$
\varphi_{f_h}^{(j)}(u) = 
\frac{
	%				\displaystyle
	F_h^{(j)}(u)-u^{(j)}
}%
{%				
	\displaystyle
	\int_{u^{(j)}}^{F_h^{(j)}(u)}
	\frac{dz}{a_j(u)z + b_j(u^{(-j)})}
}.
$$
If $u\in E_j$ then $\varphi_{f_h}^{(j)}(u) = b_j(u^{(-j)}) = f^{j}(u)$,
so $L_{\Phi}\geq 1$ fulfills  \eqref{eqn:PhifhFbound}.
On the other hand, if $u\in E_j^c$ then
\begin{equation}\label{eqn:VarPhiEjc}
\varphi_{f_h}^{(j)}(u) =
\frac{
	(F_h^{(j)}(u)-u^{(j)}) a_j(u)
}
{
	\underbrace{
		\ln \left(
		a_j(u) F_h^{(j)}(u) + b_j(u^{(-j)})
		\right)
	}_{:=R_1}
	-
	\ln \left(
	a_j(u) u^{(j)} + b_j(u^{(-j)})
	\right)
}=\Phi_j(u)f^j(u),		
\end{equation}
where
\begin{equation}\label{eqn:Ter1Simplification}	
R_1=
\ln\left\{
a_j(u)\left[
e^{ha_j(u)}u^{(j)} +
h\Phi_j(u) b_j(u^{(-j)})	
\right]
+b_j\left(u^{(-j)}\right)
\right\} \notag \\
=h a_j(u) + \ln \left( f^{(j)}(u) \right).		
\end{equation}
By lemma \ref{l1}, inequality \eqref{eqn:PhifhFbound} is satisfied 
for all $u\in E_j \cup E_j^c$. 	As $g_h(x)=g\left(F_h(x) \right)$ 
by \Cref{ass:OSLC}, then
\begin{equation} \label{eqn:gLocalLipschitzArg} 
|g_h(u)-g_h(v)|^2 \leq
L_g|F_h(u)- F_h(v)|^2  \leq
2L_g \underbrace{
	|A^{(1)}u - A^{(1)}v|^2 
}_{:=R_2} +
2L_g \underbrace{
	|A^{(2)}b(u) - A^{(2)}b(v) |^2 
}_{:=R_3}.			
\end{equation}
Let us consider each term of the right hand of inequality \eqref{eqn:gLocalLipschitzArg}.
First, note that $A^{(1)}$ is a continuous differentiable function on all 
$\R^d$, so using the mean value theorem,
we have
\begin{equation}\label{eqn:BoundTer1gh}
R_2
\leq
L_{A^{(1)}}|u-v|^2, \quad u,v \in \R^d, \quad |u|\vee |v|\leq n,
\end{equation}
for a positive constant $L_{A^{(1)}} \geq \sup_{0\leq t \leq 1} 
|\partial A^{(1)}(h, u+t(v-u))|^2$. Meanwhile, 
\begin{eqnarray}
R_3 &=&\sum_{j=1}^d\Big[\1{E_j^c}(u) \Phi_j(u) b_j(u^{(-j)}) + h \1{E_j}(u) b_j(u^{(-j)}) 
-\1{E_j^c}(v) \Phi_j(v) b_j(u^{(-j)})\nonumber \\ &-& h \1{E_j}(v) b_j(v^{(-j)})
\Big]^2 
\leq
4\sum_{j=1}^d
\Big[
\left(	
\1{E_j^c}(u) L_{\Phi} b_j(u^{(-j)})  
\right)^2
+
\left(
h \1{E_j}(u) b_j(u^{(-j)})
\right)^2
\nonumber\\&+&
\left(
\1{E_j^c}(v) L_{\Phi} b_j(v^{(-j)})
\right)^2
+
\left(
h \1{E_j}(v) b_j(v^{(-j)}
\right)^2
\Big].
\label{eqn:BoundArgTer2}	
\end{eqnarray}	
Since  $b^2_j(\cdot)$ is a function of class $C^1(\R^d)$, there is a 
constant $L_b=L_b(n)$ such that
\begin{dmath}[label={eqn:Boundbju}]
	|b_j(u)|^2 \leq L_b 
	\condition{
		$\forall u \in \R^d$,
		\quad $|u| \vee |v| \leq n$,}		
\end{dmath}
for each $j \in \{1,\cdots, d\}$. Using this bound in  \eqref{eqn:BoundArgTer2}, we obtain
\begin{equation}\label{eqn:BoundL0Ter2gh}
R_3\leq
4 \sum_{j=1}^d
\left[
2 L_{\Phi} L_b +2h^2 L_b
\right] 
\leq L_0, \qquad
\forall u,v \in \R^d,
\quad |u|\vee|v| \leq n,
\end{equation}
where $L_0=8d L_b(n)(L_{\Phi}+h^2)$.
By inequalities \eqref{eqn:BoundTer1gh} and \eqref{eqn:BoundL0Ter2gh},  we get
\begin{equation}
|g_h(u) - g_h(v)|^2 \leq L_{g_h}(n) |u - v|^2, 
\qquad	\forall u,v \in \R^d,
\quad |u|\vee |v| \leq n,		
\end{equation}
where $L_{g_h}(n)\geq n^2+1+L_0+ L_{A^{(1)}}$. Then $g_h(\cdot)$ is a locally 
Lipschitz function. Furthermore, note that under some modifications
this argument can be used to prove that $F_h(\cdot)$ is also a locally Lipschitz 
function, which implies that $\varphi_{f_{h}}$ is a locally Lipschitz function.
Finally, we  will demonstrate inequality \eqref{eqn:h-MonotoneCondition}. 
By \Cref{ass:OSLC,ass:ajBound}, we have
\begin{equation*}\label{eqn:bDotProdBound}
	\innerprod{f(u)}{u}
	=
	\sum_{j=1}^d
	a_j(u) \left( u^{(j)} \right)^2
	+
	\sum_{j=1}^d
	b_j(u) u^{(j)}
	\leq \alpha +\beta |u|^2,
\end{equation*}
and
\begin{equation*}\label{eqn:bDotProdBound}
	\innerprod{b(u)}{u} \leq \alpha + (\beta + L_a)|u|^2.
\end{equation*}
Using these inequalities and \eqref{eqn:PhifhFbound}, we deduce that
\begin{equation}\label{eqn:VarphifhDotProdBound}
	\innerprod{\varphi_{f_h}(u)}{u} 
		=\sum_{j=1}^d
		\Phi_j(u) f^{(j)}(u) u^{(j)}
		\leq L_{\Phi} L_a|u| +  L_{\Phi}(\alpha + (L_a + \beta)|u|^2) \leq
		L_{\varphi_{f_h}} (1+|u|^2).					
\end{equation}
where
	$
		L_{\varphi_{f_h}}\geq 2 L_{\Phi}\max\{L_a, \alpha, \beta\} + 1.
	$ 
Meanwhile, $g$ is globally Lipschitz then
\begin{equation}\label{eqn:GloballyLipg}
	|g_h(u)|^2 
	\leq
		2 |g(F_h(u)) - g(F_h(0))|^2  + 2 | g(F_h(0))|^2 \leq
		4 L_g |F_h(u)|^2  + 8 L_g |F_h(0)|^2  + 4 |g(0)|^2 .
\end{equation}
Now, we bound each term on the right-hand side of  \eqref{eqn:GloballyLipg}.
By the monotone condition \eqref{eqn:MonotoneCondition}, $|g(0)|^2 \leq 2\alpha$.
Moreover,
\begin{equation*}
	|F_h^{(j)} (0) |=
		h\Phi_j(0)\,| b_j(0)| \1{E_j^c}(0) + 
		h\,|b_j(0)| \1{E_j}(0)
		\leq
		\frac{b_0^*}{a_0^*}
		e^{h L_a} (1+h),
	\quad
	\forall j \in \{1, \cdots, d\}.
\end{equation*}
where
	$a^*_0 := 
	\min_{
		\substack{
			j \in \{1, \cdots, d \}\\
			a_j(0) \neq 0
		}
	}
	\left\{
		|a_j(0)|
	\right\}$
and 
	$b^*_0 :=
		\max_{\substack{
		j\in \{1,\cdots, d\}}
	}
	\left\{
		|b_j(0)|
	\right\}.$
Then
\begin{equation} \label{eqn:BoundFhZero}
	|F_h(0)|^2 
	\leq
		d\left(
			\frac{b_0^*}{a_0^*}
		\right)^2
		e^{2h L_a} (1+h)^2.
\end{equation}
Since $\Phi_j$ is bounded, from \eqref{r1} we get
\begin{dmath*}
	F_h^{(j)}(u) 
	\leq
	e^{h a_j(u)} |u^{(j)}| +
	h L_{\Phi} |b_j(u)| \1{E_j^c}(u) +h |b_j(u)| \1{E_j}(u).
\end{dmath*}
And by \Cref{ass:ajBound},
\begin{equation}\label{eqn:BoundFhu}
	|F_h^{(j)}(u)|^2 
		\leq 3 e^{2h L_a }|u|^2 + (3 h^2 L_{\Phi}^2 L_b + 3h^ 2L_b) (1+|u|^2) 
		\leq L_F(1+|u|^2), 
\end{equation}
where $L_F\geq 3 d \max\{e^{2h L_a},  h^2 L_b(L_{\Phi}^2+1)\}$.
Using \eqref{eqn:BoundFhZero} and \eqref{eqn:BoundFhu} in  inequality \eqref{eqn:GloballyLipg} yields
\begin{equation*}
	|g_h(u)|^2 \leq
		4 L_g L_F(1+|u|^2)
		+ 8 L_g d
		\left(
			\frac{b_0^*}{a_0^*}
		\right)^2
		e^{2h L_a} (1+h)^2
		+8 \alpha.
\end{equation*}
Therefore, if
$
	L_{g_h} 
	\geq 
	4 L_g L_F + 8 L_g d
	\left(
		\frac{b_0^*}{a_0^*}
	\right)^2
	e^{2h L_a} (1+h)^2
	+8 \alpha		  
$
then
\begin{equation}\label{eqn:Boundghu}
	|g_h(u)|^2
	\leq
		L_{g_h}(1+|u|^2).		
\end{equation}
Hence, from inequalities \eqref{eqn:VarphifhDotProdBound} and \eqref{eqn:Boundghu} 
and taking for each fixed $h>0$, $\alpha^* := L_{\varphi_{f_h}}\vee L_{g_h}$ and
$\beta^* := 2\alpha^*$, we obtain inequality  \eqref{eqn:h-MonotoneCondition}.
\end{proof}
%
\begin{remark}\label{rmk:PertrubedSDE}
	Note that if $b_j=0$, then \Cref{ass:ajBound} and \Cref{ass:HypThmSingularities} are unnecessary to 
	prove \Cref{lem:PhiFhProp}which is the case for stochastic Lotka-Volterra systems \cite{Mao2002,Mao2003},  
	the Ginzburg-Landau SDE \cite{Kloeden1992} or the damped Langevin equations where the potential lacks of a 
	constant term \cite{Hutzenthaler2012a}. On the other hand, there are several applications with $b_j\neq 0$ 
	among others the stochastic SIR \cite{Tornatore2005},the noisy Duffing-Van der Pol oscillator 
	\cite{Schenk-Hoppe1996b} and the stochastic Lorenz equation \cite{Gao2002}.
\end{remark}
\begin{remark}\label{rmk:PertrubedSDE}
	Note that by \Cref{lem:PhiFhProp}, we have that 
	$\displaystyle\lim_{h\to 0}|f(x)-\varphi_{f_h}(x)|=0$.
	Hence it is convenient to  consider the following modified SDE
	\begin{equation*} % \label{eqn:SDEMod}
		dy_h(t)= \varphi_{f_h}(y_h(t))dt +g_h(y_h(t))dW(t),
		\qquad y_h(0)=y_0,  \qquad t\in [0,T],
	\end{equation*}
	as a perturbation of SDE \eqref{eqn:SDE1}. 
	Moreover, the functions $\varphi_{f_h}(\cdot)$ and $g_h(\cdot)$ in \eqref{eqn:FunctionshDefinition} are
	respectively defined as the functions $\varphi_{f}$ and $g$, but  evaluated in the solution of $c=d + h\varphi(c)$, 
	then we can rewrite the \SM method \crefrange{eqn:SSLSM1}{eqn:SSLSM2} as
	\begin{align*}
		Y_k^{\star} &= Y_k + h \varphi_{f_h}(Y_k),\\
		Y_{k+1} &= Y_k^{\star} + g_h(Y_k)\Delta W_k.
	\end{align*}
	%that is, the EM method for this modified SDE.
	We formalize these ideas in the following sections.
\end{remark}
%
