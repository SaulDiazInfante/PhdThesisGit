\documentclass[review]{elsarticle}
\usepackage{amsmath,amssymb}
\usepackage{breqn}
%\usepackage{epsfig}
\usepackage{booktabs}
\usepackage{graphicx}
\usepackage[utf8]{inputenc}
\usepackage[T1]{fontenc}
\usepackage{caption}
\usepackage{siunitx}
\usepackage{hyperref}
\usepackage[labelfont=bf,textfont={sl,bf},lofdepth,lotdepth]{subfig}
%%%%%%%%%%%%%%%%%%%%%%%%%%%%%%%%%%%%%%%%%%%%%%%%%%%%%%%%%%%%%%%%%%%%%%%%%%%%%%%%%%%%%%%%%%
%                                  MODIFICACIONES                                        %
%%%%%%%%%%%%%%%%%%%%%%%%%%%%%%%%%%%%%%%%%%%%%%%%%%%%%%%%%%%%%%%%%%%%%%%%%%%%%%%%%%%%%%%%%%
\oddsidemargin=1cm
\textwidth=14cm
%%%%%%%%%%%%%%%%%%%%%%%%%%%%%%%%%%%%%%%%%%%%%%%%%%%%%%%%%%%%%%%%%%%%%%%%%%%%%%%%%%%%%%%%%%
%                                 FORMATOS                                               %
%%%%%%%%%%%%%%%%%%%%%%%%%%%%%%%%%%%%%%%%%%%%%%%%%%%%%%%%%%%%%%%%%%%%%%%%%%%%%%%%%%%%%%%%%%
\DeclareMathAlphabet{\mathpzc}{OT1}{pzc}{m}{it}
\newtheorem{dfn}{Definition}
\newtheorem{thm}{ THEOREM}[section]
\newtheorem{pro}{ PROPOSITION}[section]
\newtheorem{lem}{ LEMMA}[section]
\newtheorem{definition}{ DEFINITION}[section]
\newtheorem{corollary}{ COROLLARY}[section]
\newtheorem{consequence}{ CONSEQUENCE}[section]
\newtheorem{remark}{ REMARK}[section]
\newtheorem{example}{\bf Example}[section]
\newtheorem{proof}{\bf Proof}[section]
\newproof{pf}{Proof}

\newcommand{\fin}{\vrule height3pt width3pt depth2pt}
\newcommand{\normL}[1]{\left[\mathbb{E}\left|#1\right|^2\right]^{1/2}}
\newcommand{\ms}[1]{\mathbb{E}\left|#1\right|^2}
\newcommand{\mep}[1]{\mathbb{E}|#1|^p}
\newcommand{\m}[1]{\mathbb{E}#1}
\newcommand{\meanp}[2]{\mathbb{E}\left|#1\right|^{#2}}
\newcommand{\condexp}[2]{\mathbb{E}\left[#1|#2\right]}
\newcommand{\lftrght}[3]{\left#2 #1\right #3}\DeclareMathOperator{\tr}{tr}
\DeclareMathOperator{\diag}{diag}
%opening
\title{Important Simulation Examples for the Stochastic Steklov Scheme}
\author[sj]{S. D\'{\i}az-Infante}
\ead{sauld@cimat.mx}
\author[sj]{S. Jerez}
\ead{jerez@cimat.mx}

\begin{document}
	\begin{frontmatter}
		\title{Convergence and Asymptotic Stability  of  the Explicit
		Steklov Method for Stochastic Differential Equations 
		\tnoteref{t1}
		}%,t2}}
		\tnotetext[t1]{This work has been partially
		supported by CONACYT project *****
		}
		\author[sj]{S. D\'{\i}az-Infante}
		\ead{sauld@cimat.mx}
		\author[sj]{S. Jerez}
		\ead{jerez@cimat.mx}
		%\context[cor1]{Corresponding author. Tel.: +52 473 732 7155; fax: +52 473 732 5749.}
		\address[sj]{
		Department of Applied Mathematics, CIMAT, Guanajuato, Gto., Mexico,
		36240.
		}
		\begin{abstract}
			In this document, we develop a set of examples where the Steklov Schemes can be applied
			with good results. The examples was taken from common and variated application areas.
		\end{abstract}
		\begin{keyword}
			stochastic differential equations, Steklov average,
			explicit methods, convergence, asymptotic stability,
		\end{keyword}
	\end{frontmatter}

	\section{Steklov Scheme Construction for SDE Systems }
		Consider the vector SDE
		\begin{equation}\label{eqn:SDESystem}
			dX_t = f(X_t)dt + g(X_t)dW_t, \qquad X_0 = x_0,
		\end{equation}
		where $f:\mathbb{R}^d \to \mathbb{R}^d$, $g:\mathbb{R}^d \to \mathbb{R}^{d \times m}$, 
		$W$ is a standandard $m$-dimensional Brownian motion.
		Assuming that $f =(f^{(1)},\dots f^{(d)})^T$ and $g = (g^{(1)}, \dots g^{(m)}) $ satisfy
		the usual conditions for existence and uniqueness of solutions to \eqref{eqn:SDESystem}, 
		and 
		$H: U\subset \mathbb{R} \to \mathbb{R}$ defined by
		\begin{equation}
			H^j(u) = \int \frac{du}{
				f(X^{(1)}_{0}, \dots, X^{(j-1)}_{0}, u, X^{{(j+1)}}_{0}, \dots,  X^{(d)}_{0})
			}
		\end{equation}
		is integrable for all fixed $X_0 \in U$, and have inverse $[H^j]^{-1}$. We will construct a 
		Steklov shcme as follows.
		
		For each component equation solves the associated determinisitc problem
		$$
			\frac{dx^{(j)}}{dt} = f^{(j)}(x), \qquad X_0= x_0^j, \qquad j = 1, \dots, d
		$$
		with the following scheme
		\begin{align}
			\frac{x^{(j)}_{k+1} - x^{(j)}_{k} }{h}	&= \varphi_{f^{(j)}}(x^{(j)}_{k}, x^{(j)}_{k+1})	\\
			\varphi_{f^{(j)}} \left(x^{(j)}_{k}, x^{(j)}_{k+1} \right) &= 
				\frac{x^{(j)}_{k+1} - x^{(j)}_{k} }{H(x^{(j)}_{k+1}) -H(x^{(j)}_{k})},	\\
				H^j(u) &= 
					\int 
						\frac{du}{
							f(x^{(1)}_{k+1}, \dots, x^{(j-1)}_{k+1}, u, x^{(j+1)}_{k}, \dots, x^{(d)}_{k})
						},
		\end{align}
		here, $x^{{(1)}}_{k+1}, \dots, x^{{(j-1)}}_{k+1}$ are calculated from the first
		$j-1$ schemes. In this way we get the explicit recurrence for each $j\in \{ 1,\dots d\}$
		\begin{equation}
			x_{k+1}^{j} = 
				[H^{(j)}]^{-1}
				\left(
					h + H^{(j)}\left(x_k^{(j)}\right)
				\right).
		\end{equation}
		Finally add the approximation of the diffussion term with a Euler-Maruyama scheme to get
		\begin{equation}
			X_{k+1}^{j} = 
				[H^{(j)}]^{-1}
				\left(
					h + H^{(j)}
					\left(
						X_k^{(j)}
					\right)
				\right)
			+ g \left(X_k^{(j)} \right) \Delta W_k^{(j)}.
		\end{equation}
	\section{Alternative Construction}
		Now suppose that for ecah $j \in \{1, \dots, d \}$ there is functions 
		$a_1^{(j)}:\mathbb{R}^{d} \to \mathbb{R}$,\qquad 
		$a_2^{(j)}:\mathbb{R}^{d-1} \to \mathbb{R}$ such that 
		$$
			f^{(j)}(y) = a_1^{(j)}(y)y^{(j)} + a_2^{(j)}(y^{(-j)}), \qquad
			y^{(-j)} = (y^{(1)}, \dots ,y^{(j-1)}, y^{(j+1)}, \dots, y^{(d)}).
		$$
		Under this assumptions we propose the following scheme. For each component equation let
		\begin{align*}
			H^{(j)}(u)	&= \int \frac{du}{a_1^{(j)} u + a_2^{(j)}} 
				= \ln \left( a_1^{(j)} u + a_2^{(j)} \right),\\
			\left[H^{(j)}\right]^{(-1)} (v) & = \frac{\exp(a_1 v ) - a_2}{a_1}\\
			a_1^{(j)} &=
				a_1^{(j)}
					\left(
						x^{(1)}_{k+1}, \dots, x^{(j-1)}_{k+1},
						x^{(j)}_{k} , x^{(j+1)}_{k},\dots,x^{(d)}_{k}
					\right),
					\\
			a_2^{(j)} &=
				a_2^{(j)}
					\left(
						x^{(1)}_{k+1}, \dots, x^{(j-1)}_{k+1},
						x^{(j+1)}_{k},\dots,x^{(d)}_{k}
					\right).
		\end{align*}
		With this notation, and using the scheme (*) follows
		\begin{align}
			x_{k+1} &= \left[ H^{(j)}\right ]^{(-1)} ( H^{(j)}(x_k) + h ) \notag	\\
					&=
					\left(
						x_k^{(j)} + 
						\frac{a_2^{(j)}}{a_1^{(j)}}
					\right) 
					\exp\left(
							a_1^{(j)}h
						\right) -\frac{a_2^{(j)}}{a_1^{(j)}}.
		\end{align}
		\begin{example}
			Consider a stochastic Lotka Volterrra model proposal in \cite{Arnold1979}
		\end{example}
		\begin{example}
			Consider a stochastic Duffin Van Der Pol eqaution studied in \cite{Schenk-Hoppe1996a}
		\end{example}
	\section{Linear Oscillator}
		This model was studied in \cite{Melbo2004}.
		Melbø Aslaug, H. Strømmen, Higham Desmond. 
		Numerical simulation of a linear stochastic oscillator with additive noise
			\begin{align}
				dx(t) &= y(t)dt\\
				dy(t) &= -x(t)+h dW(t) \qquad h>0
			\end{align}6
		The Steklov Scheme is under construction.
	
	\section{Simplifiyed Duffing Van Der Pol Equation}
		This mode is proposal in the classic Kloeden's Book. \\ 
		Peter Kloeden \& Eckhard Platen. Numerical Solution of Stochastic Differential Equations.
		$$
			\ddot{x}+\dot{x}-(\alpha-x^2)x=\sigma \varepsilon
		$$
		\begin{align*}
			dX_t^{(1)}&= X_t^{(2)} dt\\
			dX_t^{(2)}&=
				\left\{
				X_t^{(1)}
				\left(
					\alpha-(X_t^{(1)})^2
				\right)
				-X_t^{(2)}
			\right\}dt
			+\sigma X_t^{(1)} dW_t
		\end{align*}
		The Steklov Scheme:
		\begin{align*}
			X^{(1)}_{k+1} &= X^{(1)}_{k} + h X^{(2)}_{k} \\
			X^{(2)}_{k+1} & =a\left(X^{(1)}_{k},X^{(1)}_{k+1}\right)
				\left(
					1 - \exp(-h)
				\right)
				+ \exp(-h) X^{(2)}_{k} + \sigma X^{(1)}_{k} \Delta W_k
		\end{align*}
		Gives good results.
		
	\section{Stochastic Duffin Van Der Pol Equation}
	
		\subsection{Model Proposal by Schenk-Hoppé, K. R}
			This SDE was studied in the Work \cite{Schenk-Hoppe1996a}:\\
			Schenk-Hoppé Bifurcation scenarios of the noisy duffing-van der pol oscillator
			\begin{align*}
				dX_t^{(1)}
					&= X_t^{(2)} dt\\
				dX_t^{(2)} &=
				\left\{
					\alpha X_t^{(1)} 
					- \left( X_t ^ {(1)} \right) ^ 3
					- \left( X_t ^ {(1)} \right) ^ 2 X_t ^ {(2)}
					+\beta X_t^{(2)}
				\right\}dt
				+ \left( \alpha+\sigma_1 dW_1 \right)X_t^{(1)}
				+ \sigma_2 dW_2
			\end{align*}
			The Steklov Scheme is sitl under construction.
	
		\subsection{Generalized Van der Pol Oscillator with multiplicative noise}
			This Generalized Van Der Pol was proposal by Schurz in \cite{Schurz2003}:\\
			Schurz, H. General Theorems for Numerical Approximations of Stochastic Processes on the Hilbert 
			Space $H_2([0,T], \mu, \mathbb{R}^d).$
			
			\begin{align*}
				dX_t^{(1)}	&= X_t^{(2)} dt \\
				dX_t^{(2)}	&=
					\left[
						-\omega^2 X_t^{(1)} 
						+\sigma 
						\left(
							1 - \mu_1 \left( X_t^{(1)} \right)^2
							- \mu_2 \left( X_t^{(2)} \right)^2
						\right)
						X_t^{(2)}
					\right]dt
				+\sigma X_t^{(2)} dW_t
			\end{align*}
			Right now we had constructed the following Linearized Steklov Schme:
			\begin{align*}
				a_1(x)&		:= -\omega^2 x, \qquad
				a_2(x,y)	:= \gamma (1-\mu_1 x_k x - \mu_2 y ),	\\
				X_{k+1} &= X_k + h Y_k,								\\
				Y_{k+1} &= \frac{a_1(X_k)}{a_2(X_k, Y_k)}
						\left(
							\exp(a_2(X_k, Y_k) h) - 1
						\right)
						+ \exp(a_2(X_k, Y_k) h) Y_k
			\end{align*}
			This scheme follows almost the same profile of numerical solutions of the Partial Linear
			Implicit proposal by Schurz.
	\section{Stochastic Lorenz Equation}
		We want to compare the performance of the Steklov Scheme with simulations of the Stochastic
		Lorenz System proposal in \cite{Keller1996}.
		H. Keller. Attractors and bifurcations of the stochastic Lorenz system.
		\begin{align*}
			du &= (-Au - B(u) + f)dt +C(u)dW(t)\\
			A & = 
			\begin{bmatrix}
				\sigma	&-\sigma	&0\\
				\sigma	&1			&0\\
				0		&0			&\beta
			\end{bmatrix}, 
			\qquad
			B(u) = 
			\begin{bmatrix}
				0\\
				xz\\
				-xy
			\end{bmatrix},
			\qquad
			f=
			\begin{bmatrix}
				0\\
				0\\
				-\beta(r+\sigma)
			\end{bmatrix}\\
			C(u) &: \mathbb{R} \to \mathbb{M}_{3 \times m}(\mathbb{R}), \qquad
			\|C(u)\|^2 = \tr\left(C(u)C(u)^T\right)\leq K^2(1+ \|u\|^2).
		\end{align*}
		In this work, the simulations was performed with $C(u)=\sqrt{q}u$.
		Apparently, the following Steklov scheme gives good results.
		\begin{align*}
			a_1(y) &= \sigma y, \qquad a_2(x,z) = -x (\sigma +z) \qquad 
			a_3(x,y) = x y -\beta(\sigma +r)	\\
			X_{k+1} &= \exp(-\sigma h) X_{k} 
				+\frac{a_1(Y_k)}{\sigma}
				(1-\exp(-\sigma h))				\\
			Y_{k+1} &= a_2(X_{k+1},Z_{k}) (1-\exp(-h))
				+Y_k \exp(-h)\\
			Z_{k+1} &= \frac{a_3(X_{k+1}, Y_{k+1})}{\beta} (1- \exp(-\beta h))
				+Z_{k} \exp(-\beta h)			\\
			u_{k+1} &= (X_{k+1},Y_{k+1},Z_{k+1})^T + \sqrt{q} \Delta W_k (X_{k},Y_{k},Z_{k})^T
		\end{align*}

	\section{Stochastic Lotka-Volterra}
		We are interested in model paths and stationary distribution of the Stochastic Lotka-Volterra model.
		First, considering two interact species, we will simulate an reproduce the results of a Model 
		perturbed with multiplicative white noise proposed in the Arnold's  ed. al. paper \cite{Arnold1979}:\\
		Arnold, L. Horsthemke, W. Stucki, J. W. The Influence of External Real and White Noise on the 
		LOTKA-VOLTERRA Model.
		\begin{align*}
			dX_t &= (\lambda X_t - k X_t Y_t ) dt +\sigma X_t dW_t\\
			dY_t &= (k X_t Y_t -mY_t) dt
		\end{align*}
		A recent similar models was presented in \cite{Yagi2011}.\\
		Atsushi Yagi, Ta Viet Ton.
		Dynamic of a stochastic predator–prey population
		\begin{align*}
			dx_t &= 
				\left[
					a_1(t) - b_1(t) x_t
					-\frac{c1(t) y_t}{\alpha(t) + \beta(t) x_t + \gamma(t) y_t  }
				\right] x_t dt
				+\sigma_1(t) x_t dW_t	\\
			dy_t &= 
				\left[
					-a_2(t) - b_2(t) x_t
					-\frac{c1(t) y_t}{\alpha(t) + \beta(t) x_t + \gamma(t) y_t  }
				\right] y_t dt
				+\sigma_2(t) y_t dW_t	\\
		\end{align*}
		However, neither mentioned work present Numerical simulations. Because of this we we will compare
		our Steklov Scheme with the numerical results for the following system presented in the paper
		\cite{Mao2002}:\\
		Xuerong Mao, Glenn Marion , Eric Renshaw.Environmental Brownian noise suppresses explosions in 
		population dynamics.
		\begin{align*}
			dx_1(t) &= 
				x1 (1 - x_1 + 2 x_2 )dt + \varepsilon x_1^2 d\omega_1(t), \\
			dx_2(t) &= 
				x2 (1 - 2 x_2 + 2 x_1 )dt + \varepsilon x_2^2 d\omega_2(t)	
		\end{align*}
		We get two Stklov Schemes. The first is the Matus average version and the second is 
		a Linearized version.
		\begin{align*}
			X_{k+1} &=	\frac{\exp(h a_1(Y_k)) X_k }{ a_1(Y_k) + X_k (\exp(h a_1(y_k)))}	\\
			Y_{k+1} &=	\frac{\exp(2 h a_2(X_{k+1})) Y_k }{ a_2(X_{k+1}) + Y_k( \exp(2h a_2( X_{k+1})))}
		\end{align*}

		This reference present results for the SDE  
		$$
			\diag(x_1(t),\dots, x_n(t))
			\left[
				f(x)dt + g(x)d\omega (t)
			\right]
		$$
		where $\omega(t) = (\omega_1(t),\dots, \omega_m(t))^T$ is a $m-$dimensional Brownian motion,
		$f: \mathbb{R}^n_+ \to \mathbb{R}^n$ and $g: \mathbb{R}^n_+ \to \mathbb{R}^{n \times m}$.
		
		However, we want to study the same model considering more species, at least 3 and 4. In this 
		direction we will follow the work \cite{Dobrinevski2012}: \\
		Dobrinevski, Alexander, Frey, Erwin.
		Extinction in neutrally stable stochastic Lotka-Volterra models.
		Physical Review E.
		In this paper, the author develope a SDE whose statistical average is the same that the deterministic
		associated Lotka Volterra system. 
		However the formulation in my opinion is not clearly enough.
		
	\section{Stochastic SIS}
		This stochastic Model was proposal in \cite{Gray2011}:
		Gray, Mao. A Stochastic Differential Equation SIS Epidemic Model
		\begin{equation}
			dI(t) = I(t)
				\left[
					\left(
						\beta N - \mu -\gamma - \beta I(t) 
					\right)
					dt
					+ \sigma (N -I(t)) dB(t)
				\right] 
		\end{equation}
		Steklov scheme:
		\begin{align*}
			X_{k+1} &= \frac{a \exp(a  h) X_k}{a + b X_k  (1 - \ exp(a h))}
			+ \sigma (N - X_k) X_k \Delta W_k\\
			a &= \beta N - \mu -\gamma\\
			b &= -\beta
		\end{align*}
	\section{Stochastic AIDS}
		We want to contruct a Stochastic Steklov Scheme for the model proposal by Dalal, Nirav
		Greenhalgh, David, Mao, Xuerong, in \cite{Dalal2007}\\
		A stochastic model of AIDS and condom use.
		\begin{align*}
			\frac{dx_1(t)}{dt} & = \lambda_1 - \mu x_1(t)		\\
			dx_2(t) & =
				\left[
					D_{21} x_4(t) - (\mu + \sigma) x_2(t)
				\right]
				dt
				- \sigma_1 x_2(t)dB(t)							\\
			\frac{dx_3(t)}{dt} &= 
				\lambda_2 - \beta c \alpha_2^2  x_4(t) \frac{x_3(t)}{T(t)} 
				- \mu x_3(t)									\\
			dx_4(t) &= 
				\left[
					\beta c \alpha_2^2  x_4(t) \frac{x_3(t)}{T(t)}
					-(\mu +\sigma + D_{21}) x_4(t)
				\right]
				- \sigma_1 x_4(t) dB(t).
		\end{align*}
		Where $T(t): = x_1(t)+ x_2(t)+ x_3(t)+ x_4(t)$.
		\\
		A most simplifiyed model is proposal in \cite{Ding2008}:
		\begin{equation*}
			dZ = 
			\left[
				(p-1) \beta Z
				+ (\beta C -\alpha )(1-Z)Z
			\right]dt
			+
			\left[
				\rho C (1-Z)Z
			\right]dW
		\end{equation*}
		Steklov Schemes are under construction.
		
		
		
	\bibliographystyle{model2-names}
	\bibliography{library}
\end{document}
