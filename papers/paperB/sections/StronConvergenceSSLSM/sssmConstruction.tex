%
	For simplicity, we begin the construction of the Linear Steklov method (\SM) considering the scalar case of SDE 
\eqref{eqn:SDE1}, that is, when $d=m=1$, also, to shorten notation we use $a,b$ instead $a_j,b_j$. 
Tough this ideas, we will generalize to higher dimensions.
Let $0=t_0 < t_1< \cdots < t_N=T$ a partition of the interval $[0,T]$ with constant step-size $h=T/N$ and such that
$t_k=kh$ for $k=0,\ldots, N$. The main idea of the  \SM approximation  consists in 
estimating the drift coefficient of \eqref{eqn:SDE1}  by
\begin{equation}
	f(y(t)) \approx 
		\varphi_{f}(y(t_{\eta_{+}(t)})) =
		\left(
			\frac{1}{y(t_{\eta_+(t)})-y(t_{\eta (t)})}
			\bigint \limits
_				{y(t_{\eta(t)})}^{y(t_{\eta_+(t)})}
					\frac{du}
						{
							a(y(t_{\eta(t)}))u
							+b
						}
	\right)^{-1}, \qquad t\in [0,T],
\end{equation}
where
\begin{align*}
	\eta(t) &:=
	k\text{  for } t\in [t_k, t_{k+1}), \quad k\geq 0,\\
	\eta_{+}(t) &:= 
	k+1  \text{ for } t\in [t_k, t_{k+1}), \quad k\geq 0.
\end{align*}
So we define the \SM method for the scalar version of the SDE \eqref{eqn:SDE1} using a split-step formulation as follows
\begin{align}
	Y_k^{\star} &= Y_k + h \varphi_f(Y^{\star}_k), \label{eqn:SSLSM1}\\
	Y_{k+1}	&= Y_k^{\star} + g(Y_k^{\star})\Delta W_k \label{eqn:SSLSM2},
\end{align}
with $Y_0=y_0$ and  $\varphi_{f}\left(Y_k^{\star}\right)$ defined by 
\begin{equation}
	\varphi_{f}\left(Y_k^{\star}\right)
	=
		\left(
			\frac{1}{Y_{k}^{\star}-Y_{k}}
			\int 
			%\limits
			_{Y_{k}}^{Y_{k}^{\star}}
				\frac{du}
				{
					a(Y_k)u
					+b
				}
	\right)^{-1}
\end{equation}
This scheme combines a split-step technique with a linear version of an exact deterministic 
method see \cite{Diaz-Infante2015,Matus2005}. 
In detail, first we compute the discrete value $Y^{\star}_k$ using the Linear Steklov approximation \eqref{eqn:SSLSM1},
and next, $Y_{k+1}$ is obtained by adding the stochastic increment $g(Y_k^\star)\Delta W_k$.

	To higher dimensions, we adapt the same split step scheme \crefrange{eqn:SSLSM1}{eqn:SSLSM2} as follows. 
For each component equation $j\in\{1, \ldots,  d\}$, on the iteration $k\in\{1, \ldots,  N\}$ take
\begin{align}
	a_{j,k} &=
	a_j
	\left(
		Y^{(1)}_{k},
		\ldots, Y^{(d)}_{k}
	\right),
	&
	b_{j,k} =
	b_{j}
	\left(
			Y^{(-j)}_k
	\right).
\end{align}
So, define  
$
	\varphi_{f}(Y_k^{\star})=
		\left(
			\varphi_{f^{(1)}}(Y_k^{\star}),
			\ldots,
			\varphi_{f^{(d)}}(Y_k^{\star})
		\right)
$
by
\begin{equation}
	\varphi_{f^{(j)}}\left(Y_k^{\star}\right)
		=
		\left(
			\frac{1}{Y_{k}^{\star(j)}-Y_{k}^{(j)}}
			\int 
			%\limits
				_{Y_{k}^{(j)}}^{Y_{k}^{\star(j)}}
				\frac{du}
				{
					a_{j,k} u
					+b_{j,k}
				}
		\right)^{-1}.
\end{equation}
%
	It is worth mentioning that even this formulation is semi implicit, we always can derive a explicit version. 
The next result deals with this issue. To simplify notation, we define  $A^{(1)}= A^{(1)}(h,u)$,  $A^{(2)}=A^{(2)}(h,u)$  and $b=b(u)$ by 
\begin{align}\label{eqn:SolutionFunctions}	
	A^{(1)}&:=
		\begin{pmatrix}
			e^{ha_1(u)} & \multicolumn{2}{c}{\text{\kern0.5em\smash{\raisebox{-1ex}{\huge 0}}}} \\
			&\ddots\\
			\multicolumn{2}{c}{\text{\kern-0.5em\smash{\raisebox{0.95ex}{\huge 0}}}} 
			& e^{ha_d(u)}
		\end{pmatrix},
		\notag
		\\
		%	
	A^{(2)}&:=
	\begin{pmatrix}
		\left(
			\displaystyle
			\frac{e^{ha_1(u)} - 1}{a_1(u)}
		\right)\1{E_1^c}	& 
		\multicolumn{2}{c}{\text{\kern0.5em\smash{\raisebox{-1ex}{\huge 0}}}}\\
		 & \ddots&\\
		\multicolumn{2}{c}{\text{\kern0.5em\smash{\raisebox{-1ex}{\huge 0}}}}&
		\left(
			\displaystyle
			\frac{e^{ha_d(u)} - 1}{a_d(u)}
		\right)\1{E_d^c}% + h \1{E_i} 
	\end{pmatrix}
	+h
	\begin{pmatrix}
		\1{E_1} & \multicolumn{2}{c}{\text{\kern0.5em\smash{\raisebox{-1ex}{\huge 0}}}}\\
		&\ddots &\\
		\multicolumn{2}{c}{\text{\kern0.5em\smash{\raisebox{-1ex}{\huge 0}}}} &
		\1{E_d}
	\end{pmatrix},\\	
	E_j&:=\{x \in \R^d: a_j(x)=0\} , \qquad 
	b(u):= \left(
		b_1(u^{(-1)}), \dots , b_d(u^{(-d)})
	\right)^T.		
	\notag
\end{align}
	Also we will need the following results from \cite[Thm 2.1]{Lawlor2012}, \cite[Thm. 1]{FineAIandKass1966}.
The first theorem will help us  with the singularities of set $E_j$ in the case where all elements of this set
are limit points. 
\begin{thm}[Multivariate L'h\^{o}pital's Rule] \label{thm:Lawlor}
	Let $\mathcal{N}$ be a neighborhood in $\R^2$ containing a point $\mathbf{q}$ at which
	two differentiable functions $f:\mathcal{N}\to \R$ and $g:\mathcal{N}\to \R$ are zero.
	Set 
	$$
		C=\{x \in \mathcal{N}: f(x)=g(x)=0 \},
	$$
	and suppose that $C$ is a smooth curve through $\mathbf{q}$.
	
	Suppose	there exist a vector $\mathbf{v}$ not tangent to $C$ at $\mathbf{q}$
	such that the directional derivative $D_{\mathbf{v}}g$ of $g$ in the direction of $\mathbf{v}$ is never zero
	within $\mathcal{N}$. Also we assume that $\mathbf{q}$ is a limit point of $\mathcal{N}\setminus C$. Then
	\begin{equation*}
		\lim_{(x,y)\to \mathbf{q}}
		\frac{f(x,y)}{g(x,y)} =
		\lim_{
				\substack{
					(x,y)\to \mathbf{q}\\ 
					(x,y)\in \mathcal{N} \setminus C
				}
		}
		\frac{D_{\mathbf{v}} f }{D_{\mathbf{v}} g}
	\end{equation*}
	if the latter limit exist.
\end{thm}
For the second auxiliary we will need the following concepts.
\begin{dfn}[Directional derivative referred at a point]
	Let $u,\mathbf{q}\in \R^2$ and $\alpha$ the positive angle respect to the $x$-axis and the segment
	$\overline{u \mathbf{q}}$.	We denote by 
	\begin{align*}
		f_{\alpha}(u) &= 
			\cos(\alpha) 		
			\frac{\partial f}{\partial u^{(1)}}(u) + 
			\sin(\alpha)
			\frac{\partial f}{\partial u^{(2)}}(u) 
			= \frac{ \innerprod{q-u}{\nabla f(u)}}{|u-q|}			
	\end{align*}
	the \emph{directional derivative respect to the point $\mathbf{q}$ on $u$}.
\end{dfn}
\begin{dfn}[Star-like set]
	A set $S\subset \R^2$ is \emph{star-like} with respect a point $\mathbf{q}$, if for each point $s \in S$ the open 
	segment $\overline{s \mathbf{q}}$ is in $S$.
\end{dfn}
%
Whit this in mine, second theorem give us a way to analyze isolated singularities.
\begin{thm}\label{thm:Fine}
	Let $\mathbf{q}\in \R^2$ and let $f$,$g$ be functions whose domains include a set $S\subset \R^2$ which is 
	star-like 
	with  respect to the point $\mathbf{q}$. Suppose that on $S$ the functions are differentiable and that
	the directional derivative of $g$ with respect to $\mathbf{q}$ is never zero. With the understanding that all 
	limits are taken from within on $S$ at $\mathbf{q}$ and if
	\begin{enumerate}[(i)]
		\item 
			$f(\mathbf{q})=g(\mathbf{q})=0$,
		\item
			$
				\displaystyle
				\lim_{x \to \mathbf{q}}
				\frac{f_{\alpha}(x)}{g_{\alpha}(x)} = L,	
			$
	\end{enumerate}
	then
	$$
		\lim_{x \to \mathbf{q}}
		\frac{f(x)}{g(x)} = L.
	$$
\end{thm}
%
With this on mind, we additionally require the following. 
\begin{hypothesis}\label{ass:HypThmSingularities}
	The set $E_j:=\{x\in \R^{d}: a_j(x)=0\}$ satisfies either:
	\begin{enumerate}[(i)]
		\item
			All point $q \in E_j$ is a non isolated zero of $a_j$ and:
			\begin{itemize}
				\item the set 
					$$
						D:=\{u \in B_r(q): e^{ha_j(u)}-1=a_j(u)= 0\},
					$$ 
					is a smooth curve through $q$. 
				\item
					The canonical vector $e_j$ is not
					tangent to $D$.
				\item
					For each $q \in E_j$, there is an open ball with center
					on $q$ and radio $r$ $B_r(q)$, such that  
					and
					$$
						a_j\neq 0, \qquad
						\frac{\partial a_j(u)}{\partial u^{(j)}} \neq 0 ,\qquad 
						\forall u \in B_r(q)
						\setminus D.
					$$	
				%				
				%				\item
				%					The limit
				%					$$
				%						\lim_{
				%							\substack{
				%								u\to u^*\\ 
				%								u \in B_r(q) \setminus D
				%							}
				%						}
				%						\frac{e^{h a_j(h)} \frac{\partial a_j(u)}{\partial u^{(j)}}}{\frac{\partial a_j(u)}{\partial 
				%						u^{(j)}}}
				%					$$
				%					
			\end{itemize}	
		\item
			All point $q \in E_j$ is a isolated zero of $a_j$ and:
			\begin{itemize}
				\item
					For each $q\in E_j$,  $q$ is not a limit point of the set 
					$E_{\alpha}:=\{x \in \R^d: (a_j)_\alpha(x)=0\}$.
				\item
					For each $q \in E_j$ there is a star-like set respect to $q$ $E_q$, such that
					the directional derivative respect to $q$ satisfies
					$$
						 (a_j)_\alpha(x) \neq 0, \qquad \forall x\in E_q.
					$$
			\end{itemize}		
	\end{enumerate}	
\end{hypothesis}
%
\begin{lem}\label{lem:PhiFhProp}
	Let \Cref{ass:OSLC,ass:ajBound,ass:HypThmSingularities} holds, and $A^{(1)}$, $A^{(2)}$, $b$  
	defined by 
	\eqref{eqn:SolutionFunctions}. Then given $u\in\mathbb{R}^d$, the equation
	\begin{equation}\label{eqn:varphiEquation}
		v = u + h \varphi_f(v),
	\end{equation}
	has a unique solution 
	\begin{equation}\label{eqn:varphiEqnSolution}
		v = A^{(1)}(h,u)u +A^{(2)}(h,u) b(u)	.
	\end{equation}
%
	If we define the functions
	$F_h(\cdot)$, $\varphi_{f_h}(\cdot)$ and $g_h(\cdot)$ by
	\begin{equation}\label{eqn:FunctionshDefinition}
		F_h(u) = v,
%			\left(
%				u + \frac{b}{a(u)}
%			\right)\exp\left(ha(u)\right)
%				-\frac{b}{a(u)},
			\qquad 
			\varphi_{f_h}(u) =\varphi_{f}(F_h(u)),
			\qquad
			g_h(u) = g(F_h(u)),
	\end{equation}
	then $F_h(\cdot)$, $\varphi_{f_h}(\cdot)$, $g_h(\cdot)$ are local Lipschitz functions 
	and for all $u\in \mathbb{R}^d$ and each $h$ fixed, there is a positive constant $L_{\Phi}$ such that
	\begin{equation}\label{eqn:PhifhFbound}
		|\varphi_{f_h}(u)|\leq L_{\Phi} |f(u)|. 
	\end{equation} 
	Moreover, for each $h$ fixed,
	%$g_h$ is a locally Lipschitz function and 
	 there are positive constants $\alpha^*$ and  $\beta^*$ such that
	\begin{equation}\label{eqn:h-MonotoneCondition}
		\innerprod{\varphi_{f_h}(u)}{u} \vee |g_h(u)|^2 \leq \alpha^* + \beta^* |u|^2, 
		\qquad
		\forall u \in \R^d.
	\end{equation}
\end{lem}
%
\begin{proof}
	%\todo{Justify the uniform convergence of  $\varphi_{f_h}$ and $g_h$.}
	Let us first, prove that \eqref{eqn:varphiEqnSolution} is solution of \Cref{eqn:varphiEquation}. To this end note 
	that for each $j\in \{1,\dots, d\}$
	\begin{equation*}
		v^{(j)} = u^{(j)} + h \varphi_{f^{(j)}}(v)	
	\end{equation*}
	So, using in \Cref{eqn:varphiEqnSolution}  that
	\begin{equation}
		\varphi_{f^{(j)}}(v) =
			\left(
				\frac{1}{v^{(j)}-u^{(j)}}
				\int 
				%\limits
				_{u^{(j)}}^{v^{(j)}}
				\frac{dz}
				{
					a_{j}(u) z
					+b_{j}(u^{(-j)})
				}
				\right)^{-1}.
	\end{equation} 
	After some algebraic manipulations we arrive at
%	
	\begin{equation}
		v^{(j)}= \exp(h a_j(u)) u^{(j)} + 
		\left[
			\left(
				\frac{\exp(ha_j(u))-1}{a_j(u)}
			\right)
			\1{E_j^c}
			+h \1{E_j}
		\right]b_{j}(u^{(-j)}),
	\end{equation}	
	for each $j\in \{1,\dots, d\}$, which is the $j$-component of the vector
	$A^{(1)}(h,u)u +A^{(2)}(h,u) b(u)$.
%From  the construction of the \SM method, we have
%	\begin{align}\label{eqn:fracLog}
%			v=u+\frac{(v - u) a(u)}{\log \left(a(u)v+b\right) - \log \left(a(u)u+b\right)}h.
%	\end{align}
%	%
%	Solving for $v$ in \eqref{eqn:fracLog} gives
%	\begin{equation*}
%		v =\left(
%			u + \frac{b}{a(u)}
%			\right)\exp\left(ha(u)\right)
%		- \frac{b}{a(u)}.
%	\end{equation*}
%	%
	Next, we will prove the inequality \eqref{eqn:PhifhFbound}. Since 
	$$
		\varphi_{f_h}^{(j)}(u) = 
			\frac{
%				\displaystyle
				F_h^{(j)}(u)-u^{(j)}
			}%
			{%				
				\displaystyle
				\int_{u^{(j)}}^{F_h^{(j)}(u)}
				\frac{dz}{a_j(u)z + b_j(u^{(-j)})}
			} \qquad.
	$$
	We first see that if $u\in E_j$, then 
	$
		\varphi_{f_h}^{(j)}(u) = b_j(u^{(-j)}) = f(u).
	$
	Then taking $C\geq 1$ we have the conclusion.
	On the other hand, if $u\in E_j^c$, then
	\begin{equation}\label{eqn:VarPhiEjc}
		\varphi_{f_h}^{(j)}(u) =
		\frac{
				(F_h^{(j)}(u)-u^{(j)}) a_j(u)
			}
			{
				\underbrace{
				\ln \left(
					a_j(u) F_h^{(j)}(u) + b_j(u^{(-j)})
				\right)
				}_{:=Ter1}
				-
				\ln \left(
					a_j(u) u^{(j)} + b_j(u^{(-j)})
				\right)
			}		
	\end{equation}
	Now consider in \Cref{eqn:VarPhiEjc} the term labeled $Ter1$, and observe that
	\begin{align}
		Ter1&=
			\ln\left\{
				a_j(u)\left[
					e^{ha_j(u)}u^{(j)} +
					\left(
						\frac{e^{ha_j(u)}-1}{a_j(u)}
					\right) b_j(u^{(-j)})	
				\right]
				+b_j\left(u^{(-j)}\right)
			\right\} \notag \\
		&=h a_j(u) + \ln \left( f^{(j)}(u) \right).
		\label{eqn:Ter1Simplification}		
	\end{align}
	Combining the relations \eqref{eqn:VarPhiEjc} and \eqref{eqn:Ter1Simplification},
	after algebraic manipulations we arrive at 
	\begin{equation}\label{eqn:VarPhiBound}
		\varphi_{f_h}^{(j)}(u) =
			\left(
				\frac{e^{ha_j(u)}-1}{h a_j(u)}
			\right)
			f^{(j)}(u), \quad \forall u \in E_j^c.
	\end{equation}
	Hence, it remains to prove that 
	\begin{equation}\label{eqn:ExpBound}
		\Phi(h, a_j)(u):=\frac{e^{ha_j(u)}-1}{ha_j(u)},
	\end{equation}
	is bounded on $\R^d$ for each $j\in \{ 1,\dots, d\}$.
	First wee see that under the \Cref{ass:OSLC} the operator $\Phi$ is continuous
	on $E_j^c$. Furthermore, for each fixed $u\in E_j^c$
	$$
		\lim_{h\to 0}
		\frac{e^{ha_j(u)}-1}{ha_j}=1.
	$$ 
	On the other hand, for each fixed $h$,  by \Cref{ass:HypThmSingularities} we obtain 
	one of the following cases:
	
	\begin{enumerate}[C{A}SE I:]
	% and \Cref{thm:} 
		\item
			\begin{equation}\label{eqn:ajSingularityCasei}
				\lim_{
					\substack{
						u \to u^*\\ 
						u\in E_j^c
					}
				}
				\Phi(h,a_j)(u) =
				%
				\lim_{
					\substack{
						u \to u^*\\ 
						u\in E_j^c
					}
				}	
				\frac{\frac{\partial a_j(u)}{\partial u^{(j)}} 
					h e^{h a_j(u)} 
				}{
					h\frac{\partial a_j(u)}{\partial u^{(j)}}
				}=1.
			\end{equation}	
		\item
			\begin{equation}\label{eqn:ajSingularityCaseii}
			\lim_{
				\substack{
					u \to u^*\\ 
					u\in E_j^c
				}
			}
			\Phi(h,a_j)(u) 
			=
			%
			\lim_{
				\substack{
					u \to u^*\\ 
					u\in E_j^c
				}
			}	
			\frac{
				\left(
					 e^{h a_j(u)} - 1
				 \right)_{\alpha}
			}{
				\left(
					h a_j(u)
				\right)_{\alpha}
			}	=	1, \qquad \alpha = 0,\pi, 2\pi,\dots
		\end{equation}	
	\end{enumerate}
	%
	%
	%
	So, under this situation, we can say for each $j$ that the function
	\begin{equation*}
		f^{(j)}(u) \1{E_j} +
		\left(
			\frac{e^{ha_j(u)}-1}{ha_j(u)}
		\right)
		f^{(j)}(u) \1{E_j^c}
	\end{equation*}
	is continuous on $\R^d$ and bounded on $E_j$.
	Now, let
		$$
			a^*_j:= \inf_{u\in E_j^c}
			\{
				|a_j(u)|
			\}.
		$$
	So, $a^*_j$ satisfy one of the two following cases:
	\begin{enumerate}[C{A}SE I:]
		\item
			$
				\displaystyle
				0 < a_j^*\leq L_a. 
			$
		\item
			$a_j^*=0$.
	\end{enumerate}
	In the first case we see that
	\begin{align*}
		\frac{e^{h a_j(u)}-1}{ha_j(u)}
		& \leq
			\frac{e^{h L_a}-1}{ha^*_j(u)} <\infty,
		\qquad \forall h\in (0,\infty ).
	\end{align*}
	For CASE II, we can apply an argument similar as in \crefrange{eqn:ajSingularityCasei}{eqn:ajSingularityCaseii}.
	Then there is  $L_{\Phi}>0$ such that
	\begin{equation}\label{eqn:PhiBound}
		\left|
			\frac{e^{ha_j(u)}-1}{ha_j(u)}
		\right| < L_{\Phi},
		\qquad \forall u \in \R^d.
	\end{equation}	
%		
	Combining this fact with \Cref{eqn:VarPhiBound}, we obtain
	$$
		|\varphi_{f_h^{(j)}}(u)|
		\leq
		\left|
			\frac{e^{ha_j(u)}-1}{ha_j(u)}
		\right||f^{(j)}(u)| < L_{\Phi} |f^{(j)}(u)|,
		\qquad \forall u \in \R^d,
	$$
	which prove inequality \eqref{eqn:PhifhFbound}.
	
		No, we prove the $g_h$ is  a locally Lipschitz function. 
	By (H-\ref{ass:C1Functions}) in \Cref{ass:OSLC},  $g$ is a globally Lipschitz function, so
	\begin{align*}
		g_h(x)
		&=
			g\left(F_h(x) \right),
	\end{align*}
	is the composition of a continuous bounded function and a globally Lipschitz function, furthermore, note that
	\begin{align} \label{eqn:gLocalLipschitzArg} 
		|g_h(u)-g_h(v)|^2 
		&\leq
			L_g|F_h(u)- F_h(v)|^2 \notag  \\
		&\leq
			2L_g \underbrace{
				|A^{(1)}(h,u)u - A^{(1)}(h,v)v|^2 
			}_{:=Ter_1} +
			2L_g \underbrace{
				|A^{(2)}(h,u)b(u) - A^{(2)}(h,v)b(v) |^2 
			}_{:=Ter_2}.			
	\end{align}
	Now, we work with each term of the right hand of inequality \eqref{eqn:gLocalLipschitzArg}.
	First note that $A^{(1)}$ is a continuous differentiable function on all $\R^d$ so by the mean value Theorem 
	we have
	\begin{equation*}
		|A^{(1)}(h,u)u - A^{(1)}(h,v)v|^2 
		\leq
		\sup_{0\leq t \leq 1} 
		|\partial A^{(1)}(h, u+t(v-u))|^2 |u-v|^2,	
	\end{equation*}
	then, there is a positive constant $L_{A^{(1)}} = L_{A^{(1)}}(h,n)$ such that
	\begin{equation}\label{eqn:BoundTer1gh}
		|A^{(1)}(h,u)u - A^{(1)}(h,v)v|^2 
		\leq
		L_{A^{(1)}}|u-v|^2, \quad u,v \in \R^d, \quad |u|\vee |v|\leq n.
	\end{equation} 
	In the other hand, 
	\begin{dgroup*}
		\begin{dmath*}
			Ter_2 =
				\sum_{j=1}^d
				\left[	
					\1{E_j^c}(u) \Phi(h,a_j)(u) b_j(u^{(-j)}) + h \1{E_j}(u) b_j(u^{(-j)}) 
					-
					\1{E_j^c}(v) \Phi(h,a_j)(v) b_j(u^{(-j)}) - h \1{E_j}(v) b_j(v^{(-j)})
				\right]^2
		\end{dmath*}
		%
		\begin{dmath*}
			\leq
			4\sum_{j=1}^d
			\left[
				\left(	
					\1{E_j^c}(u) \Phi(h,a_j)(u) b_j(u^{(-j)})  
				\right)^2
				+
				\left(
					 h \1{E_j}(u) b_j(u^{(-j)})
				\right)^2
				+
				\left(
					\1{E_j^c}(v) \Phi(h,a_j)(v) b_j(v^{(-j)})
				\right)^2
				+
				\left(
					  h \1{E_j}(v) b_j(v^{(-j)}
				\right)^2
			\right]
		\end{dmath*}
		%
		\begin{dmath}[label={eqn:BoundArgTer2}]
			\leq
			4\sum_{j=1}^d
			\left[
				\left(	
					\1{E_j^c}(u) L_{\Phi} b_j(u^{(-j)})  
				\right)^2
				+
				\left(
					h \1{E_j}(u) b_j(u^{(-j)})
				\right)^2
				+
				\left(
					\1{E_j^c}(v) L_{\Phi} b_j(v^{(-j)})
				\right)^2
				+
				\left(
				 h \1{E_j}(v) b_j(v^{(-j)}
				\right)^2
			\right].
		\end{dmath}		
	\end{dgroup*}	
	Since each $b^2_j(\cdot)$ function is of class $C^1(\R^d)$, there is a constant $M_b=M_b(n)$ such that
	\begin{dmath}[label={eqn:Boundbju}]
		|b_j(u)|^2 \leq M_b 
		\condition{
			$\forall u \in \R^d$,
			\quad $|u| \vee |v| \leq n$,
			\quad $j \in \{1,\cdots, d\}$
		}.		
	\end{dmath}
	Putting this bound in inequality \eqref{eqn:BoundArgTer2}, we deduce that
	\begin{dgroup*}
		\begin{dmath*}
			Ter_2 
			\leq
			4 \sum_{j=1}^d
				\left[
					2L_{\Phi} M_b +2h^2 M_b
				\right] 
		\end{dmath*}
		\begin{dmath}[label={eqn:BoundL0Ter2gh}]
			\leq
			\underbrace{
				8d M_b(n)(L_{\Phi}+h^2)
			}_{:=L_0}
			\condition{
				$\forall u,v \in \R^d$,
				\quad $|u|\vee|v| \leq n$.	
			}
		\end{dmath}
	\end{dgroup*}
	Then, combining inequalities \eqref{eqn:BoundTer1gh} and \eqref{eqn:BoundL0Ter2gh} we arrive at
	\begin{dgroup*}
		\begin{dmath*}
			|g_h(u) - g_h(v)|^2
			\leq
			L_{A^{(1)}} |u - v|^2 + L_0.
		\end{dmath*}
		\begin{dsuspend}
			So, taking $L_{g_h}=L_{g_h}(h,n) \geq n^2+1+L_0+ L_{A^{(1)}}$, we see that 
		\end{dsuspend} 
		\begin{dmath*}
			|g_h(u) - g_h(v)|^2 \leq L_{g_h}(n) |u - v|^2 
			\condition{
				$\forall u,v \in \R^d$,
				\quad $|u|\vee |v| \leq n$.
			}
		\end{dmath*}
	\end{dgroup*}
	Then $g_h(\cdot)$ is a locally Lipschitz function. Furthermore, note that under some modifications,
	this argument also prove that $F_h(\cdot)$ is a Locally Lipschitz function, which also implies that
	$\varphi_{f_{h}}$ is a locally Lipschitz function.
	Finally, we prove the inequality \eqref{eqn:h-MonotoneCondition}. We first observe that by
	\Cref{ass:OSLC,ass:ajBound}
	\begin{dgroup*}
		\begin{dmath*}
			\innerprod{f(u)}{u}
			=
			\sum_{j=1}^d
				a_j(u) \left( u^{(j)} \right)^2
			+
			\sum_{j=1}^d
			b_j(u) u^{(j)}
			\leq \alpha +\beta |u|^2,
		\end{dmath*}
		\begin{dsuspend}
			then
		\end{dsuspend}
		\begin{dmath}[label = {eqn:bDotProdBound}]
			\innerprod{b(u)}{u} \leq \alpha + (\beta + L_a)|u|^2.
		\end{dmath}
	\end{dgroup*}
	With this on mind, and using the inequality \eqref{eqn:PhifhFbound}, we deduce that
	\begin{dgroup*}
		\begin{dmath*}
			\innerprod{\varphi_{f_h}(u)}{u} 
				=
					\sum_{j=1}^d
						\Phi(h,aj)(u) f^{(j)}(u) u^{(j)}
				\leq
					\sum_{j=1}^d
						L_{\Phi}L_a|u^{(j)}|
						+ L_{\Phi}\innerprod{b(u)}{u}
				\leq
					L_{\Phi} L_a|u| + L_{\Phi}(\alpha + (L_a + \beta)|u|^2).					
		\end{dmath*}
	\end{dgroup*}
	
	So, taking 
	$
		L_{\varphi_{f_h}} = L_{\varphi_{f_h}}(h) \geq 2 L_{\Phi} \cdot \max\{L_a, \alpha, \beta\} + 1,
	$ 
	we obtain
	\begin{equation}\label{eqn:VarphifhDotProdBound}
		\innerprod{ \varphi_{f_h}(u)}{u}
		\leq 
			L_{\varphi_{f_h}} (1+|u|^2), \quad \forall u \in \R^d.
	\end{equation}
	On the other hand, since $g$ is globally Lipschitz it follows that
	\begin{align}\label{eqn:GloballyLipg}
		|g_h(u)|^2 
		&\leq
			2 |g(F_h(u)) - g(F_h(0)|^2  + 2 | g(F_h(0)|^2 \notag \\
		&\leq 
			2 L_g |F_h(u) - F_h(0)|^2  + 4 |g(F_h(0)) - g(0)|^2 + 4 |g(0)|^2 \notag\\ 
		&\leq 
			4 L_g |F_h(u)|^2  + 8 L_g |F_h(0)|^2  + 4 |g(0)|^2 .
	\end{align}
	Now we bound each term in the right hand side of inequality \eqref{eqn:GloballyLipg}.
	Let us first observe that by the monotone condition in \Cref{ass:OSLC}
	\begin{equation} \label{eqn:Boundgzero}
		|g(0)|^2 \leq 2\alpha.
	\end{equation}
	On the other hand 
	\begin{align*}
		F_h^{(j)} (0) =
			\frac{e^{h a_j(0)}-1}{a_j(0)} b_j(0) \1{E_j^c}(0) + hb_j(0) \1{E_j}(0).
	\end{align*}
	So, taking
	\begin{align*}
		a^*_0 := 
			\min_{
				\substack{
					j \in \{1, \cdots, d \}\\
					a_j(0) \neq 0
				}
			}
			\left\{
				|a_j(0)|
			\right\},
			\qquad
		b^*_0 :=
			\max_{
				j\in {1,\cdots, d}
			}
			\left\{
				|b_j(0)|
			\right\}
	\end{align*}
	we can deduce that
	\begin{equation*}
		|F_h^{(j)}(0)| 
			\leq
			\frac{b_0^*}{a_0^*}
			e^{h L_a} (1+h),
			\quad
			\forall j \in \{1, \cdots, d\}.  
	\end{equation*}
	Then
	\begin{equation} \label{eqn:BoundFhZero}
		|F_h(0)|^2 
			\leq
			d\left(
				\frac{b_0^*}{a_0^*}
			\right)^2
		e^{2h L_a} (1+h)^2.
	\end{equation}
	In this line, since the operator $\Phi(h,a_j)$ is bounded, it follows that
	\begin{dmath*}
		F_h^{(j)}(u) 
			=
			e^{h a_j(u)} u^{(j)} +
			h \Phi(h, a_j)(u) b_j(u) \1{E_j^c}(u) +h b_j(u) \1{E_j}(u)
			\leq
			e^{h a_j(u)} |u^{(j)}| +
			h L_{\Phi} |b_j(u)| \1{E_j^c}(u) +h |b_j(u)| \1{E_j}(u).
	\end{dmath*}
	Then we deduce by \Cref{ass:ajBound} that
	\begin{align*}
		|F_h^{(j)}(u)|^2 
		&\leq
			3 e^{2h L_a }|u|^2 + (3 h^2 L_{\Phi}^2 L_b + 3h^ 2L_b) (1+|u|^2) \\
		&\leq
			3 h^2 L_b (1 + L_{\Phi}^2)   +
			3 \left(
				 e^{2 h L_a} + h^2 L_b (L_{\Phi}^2 + 1 ) 
			\right)|u|^2.	
	\end{align*}
	So, taking 
	$
		L_F\geq 3 d \max\{ \exp(2h L_a),  h^2 L_b(L_{\Phi}^2+1)\}
	$,
	we obtain that
	\begin{equation}\label{eqn:BoundFhu}
		|F_h^{(j)}(u)|^2
		\leq
			L_F(1+|u|^2).	
	\end{equation}
	Then, combining bounds \eqref{eqn:Boundgzero},\eqref{eqn:BoundFhZero} and \eqref{eqn:BoundFhu} we arrive at
	\begin{equation*}
		|g_h(u)|^2 \leq
			4 L_g L_F(1+|u|^2)
			+ 8 L_g d
			\left(
				\frac{b_0^*}{a_0^*}
			\right)^2
			e^{2h L_a} (1+h)^2
			+8 \alpha.
	\end{equation*}
	Therefore, if
	$
		 L_{g_h} 
		 \geq 
			  4 L_g L_F + 8 L_g d
			  \left(
				  \frac{b_0^*}{a_0^*}
			  \right)^2
			  e^{2h L_a} (1+h)^2
			  +8 \alpha		  
	$
	wee see that
	\begin{equation}\label{eqn:Boundghu}
		|g_h(u)|^2
		\leq
			L_{g_h}(1+|u|^2)		
	\end{equation}
	Hence, from the inequalities \eqref{eqn:VarphifhDotProdBound} and \eqref{eqn:Boundghu} and taking
	for each fixed $h>0$
	$$
		\alpha^* := L_{\varphi_{f_h}}\vee L_{g_h}, \qquad
		\beta^* := 2\alpha^*
	$$ 
	 we have the desired conclusion.\qed
\end{proof}
%
\begin{remark}\label{rmk:PertrubedSDE}
	Note that if $b_j=0$ in (A-1) 
	then Hypotheses \ref{ass:ajBound} and \ref{ass:HypThmSingularities} are unnecessary to 
	prove \Cref{lem:PhiFhProp}. Several  applications  as some stochastic
	Lotka-Volterra systems \cite{Mao2002,Mao2003},  
	the Ginzburg-Landau SDE \cite{Kloeden1992} or
	the damped Langevin Equations where the potential lacks of a constant term 
	\cite{Hutzenthaler2012a} have this form. 
	By other hand, if $bj\neq 0$ then SDE as 
	the stochastic SIR \cite{Tornatore2005},
	the noisy Duffing-Van der Pol Oscillator \cite{Schenk-Hoppe1996b}, the stochastic Lorenz equation
	\cite{Gao2002}, among others follow this structure.
\end{remark}
\begin{remark}\label{rmk:PertrubedSDE}
	Note that by \Cref{lem:PhiFhProp}, we have that 
	$\displaystyle\lim_{h\to 0}|f(x)-\varphi_{f_h}(x)|=0$.
	Hence it is convenient to  consider the following modified SDE
	\begin{equation*} % \label{eqn:SDEMod}
		dy_h(t)= \varphi_{f_h}(y_h(t))dt +g_h(y_h(t))dW(t),
		\qquad y_h(0)=y_0,  \qquad t\in [0,T],
	\end{equation*}
	as a perturbation of SDE \eqref{eqn:SDE1}. 
	Moreover, the functions $\varphi_{f_h}(\cdot)$ and $g_h(\cdot)$ in \eqref{eqn:FunctionshDefinition} are
	respectively defined as the functions $\varphi_{f}$ and $g$, but  evaluated in the solution of $c=d + h\varphi(c)$, 
	then we can rewrite the \SM method \crefrange{eqn:SSLSM1}{eqn:SSLSM2} as
	\begin{align*}
		Y_k^{\star} &= Y_k + h \varphi_{f_h}(Y_k),\\
		Y_{k+1} &= Y_k^{\star} + g_h(Y_k)\Delta W_k.
	\end{align*}
	%that is, the EM method for this modified SDE.
	We formalize these ideas in the following sections.
\end{remark}
%
