
	In this section, we show that, it is possible to establish a rate of convergence for the \SM
method in compact domains. Here we have followed  the same stopping time technique as the employed by 
\citeauthor{Mao2013} in \cite[Lem. 4.3]{Mao2013}.
In the following result, we state in \Cref{thm:StrongConvergenceOrder}, that  the \SM approximation converges strongly 
to the true solution with the standard convergence order one half, at least on compact domains.

\begin{thm}\label{thm:StrongConvergenceOrder}
	Let \Cref{ass:OSLC} holds and $\theta_R$ defined as in \eqref{eqn:StoppingTimes}. Then for sufficiently large $R>0$ 
	there exist a positive constant $C=C(T,R)$ such that
	\begin{equation*}
	\EX{
			\sup_{0 \leq t \leq T}
				\left|
					\overline{Y}(t\wedge \theta_R)
						-y(t\wedge \theta_R)
				\right|^2
		}
		\leq
		C(T,R)h .
	\end{equation*}
	\end{thm}
%
\begin{pf}
		The proof is based on the estimation of three convenient integrals.
	By the It\^{o} isometry and other standard 
	inequalities we get 
	\begin{align}
		\underbrace{
		\EX{
			\sup_{0 \leq t \leq T_1}
			\left|
			\overline{Y}(t\wedge \theta_R)
			- y(t\wedge \theta_R)
			\right|^2
		}}_{:=LHS}
		&\leq
		2 \EX{
			\sup_{0\leq t\leq T}
			\left|
			\int_{0}^{t\wedge \theta_R}
			\varphi_{f_h}(Y_{\eta(s)})
			-f(y(s))ds
			\right|^2 
		}\notag \\
		&+2 \EX{
			\sup_{0\leq t\leq T}
			\left|
			\int_{0}^{t\wedge \theta_R}
			g_h(Y_{\eta(s)})
			-g(y(s))ds
			\right|^2
		}, \notag
	\end{align}
	for any $T_1\in[0,T]$ and fixed $R>0$.
		%
	Now, using the H\"older and Buckholder-Davis-Gundy inequalities, we have
	\begin{align}
		LHS&\leq
			2(t\wedge \theta_R)
			\EX{
			\sup_{0\leq t\leq T}
			\int_{0}^{\theta_R \wedge t}
			\left|
			\varphi_{f_h}(Y_{\eta(s)})
			-f(y(s))
			\right|^2 ds 
		}\notag \\
		&+
		8\EX{
			\int_{0}^{\theta_R \wedge t}
			\left|
			g_h(Y_{\eta(s)})
			-g(y(s))
			\right|^2 ds
		}.\notag
	\end{align}
		%
		%
	Next, applying \Cref{lem:PhiFhProp} and since $\int_0^t |\cdot|ds$ is non decreasing, we gets
	\begin{align}
		LHS
		&\leq
		2T
		\EX{
			\int_0^{T_1\wedge \theta_R}
			\left|
			\varphi_{f}(Y^\star_{\eta(s)})
			-f(y(s))
			\right|^2 ds 
		}\notag \\
		&+
		8\EX{
			\int_{0}^{\theta_R \wedge t}
			\left|
			g_h(Y_{\eta(s)})
			-g(y(s))
			\right|^2 ds
		}.\label{eqn:BoundLocalConvStep1} 
	\end{align}
		%
	By \Cref{lem:PhiFhProp} we deduce that
	\begin{align*}
		\left|
			\varphi_{f}(Y^\star_{\eta(s)})
		-f(y(s))
		\right|^2
		&=
		\left|
		Y_{\eta(s)}
		\frac{
			\left(\exp(h a_1(Y_{\eta(s)}))-1\right)
		}{h}
		-f(y(s))
		\right|^2 \notag \\
		&\leq
		2|f(Y_{\eta(s)})-f(y(s))|^2 +
		\mathcal{O}(h^2).
	\end{align*}
%
	Combining this bound with the inequality \eqref{eqn:BoundLocalConvStep1} and since by hypothesis $f,g$ are locally 
	Lipschitz functions, we obtain
	\begin{align}
		LHS
		&\leq 
		4TC(R)
		\EX{
			\int_0^{T_1\wedge \theta_R}
			\left|
			Y_{\eta(s)})
			-y(s)
			\right|^2 ds 
		}
		+C(R)4Th^2
		\notag \\
		&+
		8C(R)\EX{
			\int_{0}^{\theta_R \wedge t}
			\left|
			Y^{\star}_{\eta(s)}
			-y(s)
			\right|^2 ds
		}.\label{eqn:IntBoundByLipschitz}
	\end{align}
%
%
	Adding conveniently zeros, we can rewrite \eqref{eqn:IntBoundByLipschitz} as 
	\begin{align}
		LHS
		&\leq
		8TC(R)
		\EX{
			\int_0^{T_1\wedge \theta_R}
			\left|
				Y_{\eta(s)})
				-\overline{Y}(s)
			\right|^2 ds 
			+
			\int_0^{T_1 \wedge \theta_R}
			\left|
			\overline{Y}(s)
			-y(s)
			\right|^2 ds 
		}
		\notag \\
		&+
		16C(R)\EX{
			\int_{0}^{\theta_R \wedge t}
			\left|
				Y^{\star}_{\eta(s)}
				-\overline{Y}(s)
			\right|^2 ds
			+\int_0^{T_1\wedge \theta_R}
			\left|
			\overline{Y}(s)
			-y(s)
			\right|^2 ds
		}
		+ C(R) 4 T h^2
		\notag \\
		&\leq
		8C(R)
		\underbrace{
			\EX{
				\int_0^{T_1\wedge \theta_R}
				\left|
				Y_{\eta(s)}
				-\overline{Y}(s)
				\right|^2 ds
			}
		}_{:=I_1}
		%
		+
		\underbrace{
			\EX{
				\int_0^{T_1\wedge \theta_R}
				\left|
				Y^{\star}_{\eta(s)}
				-\overline{Y}(s)
				\right|^2 ds
			}
		}_{:=I_2}
		\notag\\
		&+
		8C(R)(T+2)
		\EX{
			\int_0^{T_1\wedge \theta_R}
			\left|
			\overline{Y}(s)
			-y(s)
			\right|^2 ds
		}.
		\label{eqn:BoundLocalConvStep2}
	\end{align}
	%
	Now, we bound each integral of the inequality \eqref{eqn:BoundLocalConvStep2}. First, we begin with $I_1$. 
	Let $s\in [0, T_1\wedge \theta_R]$ and using the inequality 
	\eqref{eqn:SingleHolder}, we get
	\begin{align}
		|Y_{\eta(s)} -\overline{Y}(s)|^2
		&\leq
		2\left\{
		h^2 |\varphi_{f_h}(Y_{\eta(s)})|^2
		+
		|g_h(Y_{\eta(s)})|^2
		|W(s)-W(\eta(s))|^2
		%
		\right\}.\label{eqn:DifferenceBetweenDisConParts}
	\end{align}
		%
	By \Cref{lem:PhiFhProp}, we know that $\varphi_{f_h}$ and $g_h$ are locally Lipschitz functions, so
	there is a positive constant $C(R)>0$ such that 
	\begin{align}
		|\varphi_{f_h}(x)|^2
		&\leq
		2\left(
		\left|
		\varphi_{f_h}(x)
		-\varphi_{f_h}(0)
		\right|^2
		+\left|
		\varphi_{f_h}(0)
		\right|^2
		\right)
		\leq 
		2C(R)|x|^2,
		\quad 
		\\
		%
		|g_h(x)|^2
		&\leq
		2\left(
		\left|
		g_h(x)
		-g_h(0)
		\right|^2
		+\left|
		g_h(0)
		\right|^2
		\right)
		\leq 
		2\left(
		C(R)|x|^2+|g_h(0)|^2
		\right),\label{eqn:ghBound}
		\end{align}
	for all $x \in \R$ and $|x|\leq R$.
	Using both of these bounds in the inequality  \eqref{eqn:DifferenceBetweenDisConParts}, we obtain
	\begin{align*}
		|Y_{\eta(s)}-\overline{Y}(s)|^2
		&\leq
		4h^2 C(R)|Y_{\eta(s)}|^2
		+
		2\left(
		C(R)|Y_{\eta(s)}|^2
		+|g_h(0)|^2
		|W_s-W_{\eta(s)}|^2
		\right).
	\end{align*}
		%
		%
	By \Cref{col:ContinuousExtBoundedMoments}, the \SM method has bounded moments, so there exist a constant $B>0$ such
	that
	\begin{align}
		\int_0^{T_1\wedge \theta_R}
		\EX{
			|Y_{\eta(s)}-\overline{Y}(s)|^2
		}ds
		&\leq
		4C(R)(h^2+h)
		\int_{0}^T
		\EX{|Y_{\eta(s\wedge \theta_R)}|^2}ds
		+4C(R)hT|g_h(0)|^2
		\notag\\
		&\leq
		4C(R)h(T+1)
		BT
		+4C(R)hT|g_h(0)|^2 \notag\\
		&\leq
		C(R,T)h. \label{eqn:BoundLocalConvStep3}
	\end{align}
		%
		%
	The next step is to show that the integral $I_2$ is bounded. 
	Using \Cref{lem:PhiFhProp},  the second order Taylor expansion of 
	$\exp(ha_1(Y_{\eta(s)}))$ and the inequality \eqref{eqn:SingleHolder}, we can deduce that
	\begin{align*}
		|Y_{\eta(s)}^{\star} - \overline{Y}(s)|^2
			&=|Y_{\eta(s)}\exp(ha_1(Y_{\eta(s)})) - \overline{Y}(s)|^2 \notag \\
			&\leq
				2 |Y_{\eta(s)} - \overline{Y}(s)|^2  + 4h^2|f(Y_{\eta(s)})|^2 +\calO(h^4),
		\quad \text{for all }s\in [0,T_1\wedge \theta_R].
	\end{align*}
	Note that $f$ is a locally Lipschitz function, so, using a similar argument as in the inequality 
	\eqref{eqn:ghBound}, we get
	\begin{equation*}
		|f(x)|^2
			\leq
				2\left(
					\left|
						f(x)-f(0)
					\right|^2
				+\left|
					f(0)
				\right|^2
			\right)
			\leq 
			2\left(
			C(R)|x|^2 %+|f(0)|^2
			\right).
	\end{equation*}
	%
	Thus
	\begin{equation}\label{eqn:AfeterfLipschitzBound}
		|Y_{\eta(s)}^{\star} - \overline{Y}(s)|^2
		\leq
		2 |Y_{\eta(s)} - \overline{Y}(s)|^2  + 4h^2C(R)|Y_{\eta(s)}| +C(R)h^2.
	\end{equation}
	%%
	Using \eqref{eqn:AfeterfLipschitzBound}, the inequality \eqref{eqn:BoundLocalConvStep3} and 
	\Cref{col:ContinuousExtBoundedMoments} 
	gives
	\begin{align}
		\int_{0}^{T_1\wedge \theta_R}
		\EX{|Y_{\eta(s)}^{\star} - \overline{Y}(s)|^2}ds
		&\leq
			2\int_{0}^{T_1\wedge \theta_R}
				\EX{|Y_{\eta(s)} - \overline{Y}(s)|^2}ds
			\notag\\
		&+
		2h(h+T)C(R)
			\int_0^T
				\EX{|Y_{\eta(s\wedge \theta_R)}|^2} 
		+
		TC(R)h^2
		\notag \\
		&\leq
		2hC(R,T) + 2h(h+T)C(R) B T  + TC(R)h
		\notag \\
		&\leq
		hC(R,T). \label{eqn:BoundLocalConvStep4}
	\end{align}
	%
	Combining the inequalities \eqref{eqn:BoundLocalConvStep3}, \eqref{eqn:BoundLocalConvStep4} and 
	\eqref{eqn:BoundLocalConvStep2}, we deduce that
		\begin{align*}
		\EX{
			\sup_{0 \leq t \leq T_1}
			\left|
			\overline{Y}(t\wedge \theta_R)
			- y(t\wedge \theta_R)
			\right|^2
		}
		&\leq
		8C(R)(T+2)
		\int_{0}^{T_1\wedge \theta_R}
		\EX{
			\left|
			\overline{Y}(s)
			-y(s)
			\right|^2
		}\notag
		\\
		&+8C(R,T)h + 16C(R,T)h\notag
		\\
		&\leq
		8C(R)(T+2)\int_0^{T_1}
		\EX{
			\sup_{0 \leq t \leq s}
			\left|
			\overline{Y}(t\wedge \theta_R)
			- y(t\wedge \theta_R)
			\right|^2ds
		}\notag \\
		&+C(R,T)h.
		\end{align*}
		%
		Finally, applying the Gronwall inequality \eqref{thm:Gronwall} we obtain
		\begin{align*}
		\EX{
			\sup_{0 \leq t \leq T_1}
			\left|
			\overline{Y}(t\wedge \theta_R)
			- y(t\wedge \theta_R)
			\right|^2
		}
		&\leq
		C(R,T)\exp\left(8C(R)(T+2)\right) h \notag
		\\
		&\leq
		C(R,T)h,
		\end{align*}
		as we require. $\square$
	\end{pf}
