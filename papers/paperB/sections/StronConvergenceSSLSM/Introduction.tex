	Applications of Monte Carlo simulations \cite{Glasserman2004,Giles2008} and  Brownian Dynamics \cite{Cruz2012}
require  fast numerical methods with low computational cost --- excluding the use of 
implicit schemes in most cases. The Euler-Maruyama method is the leader in such  
simulations due to its simple algebraic structure, a cheap computational cost and an acceptable convergence rate 
under global Lipschitz conditions.  However, if the drift or diffusion coefficients of stochastic 
differential equations (SDEs) grows faster than a linear function, then the Euler-Maruyama approximation 
diverges  in the mean square sense \cite{Hutzenthaler2009, Hutzenthaler2012b}. 
In most cases, the coefficients of stochastic  models in finances, biology or physics 
have super-linear growth and locally Lipschitz coefficients. Therefore recent research has been 
focused on modifying the Euler-Maruyama method to obtain strong convergence  under these conditions keeping
its simple structure and a low computational cost. In the last years, 
several methods have been developed in this direction:  the family of  Tamed schemes
\cite{Hutzenthaler2012a, Wang2011, Zong2014,Hutzenthaler2015, Sabanis2015}, 
a special type of balanced method \cite{Tretyakov2013},  the stopped scheme \cite{Liu2013a} and 
a truncated Euler method  \cite{Mao2015}.  In these works, the strong convergence of the new method
is proved using the theory developed by Higham, Stuart and 
Mao in \cite{Higham2002b} or by means of  the new approach given by  \citeauthor*{Hutzenthaler2015} in 
\cite{Hutzenthaler2015}.
Both techniques obtain the property of  strong convergence by proving boundedness moments of the numerical and 
analytical solution of the underlying SDE. In spite of the recent work in this subject,  it is still necessary
to construct more accurate numerical methods for SDE under super-linear growth and 
non-globally Lipschitz coefficients.

In this chapter, we develop an explicit method based on a linear version  of the Steklov method proposed 
in the last chapter (published on \cite{Diaz-Infante2015})  for the  vector It\^o stochastic differential equation:
\begin{equation}\label{eqn:SDE1}
dy(t)
=f(y(t))dt + g(y(t))dW(t), \quad 0\leq t\leq T,
\quad y(0)=y_0,
\end{equation}
where $(f^{(1)},\dots, f^{(d)}):\R^d \to \R^d$ is one sided Lipschitz and 
$g = (g^{(j,i)})_{j\in \{1,\dots,d\}, i\in\{1,\dots, m\}}:\R^d \to \R^{d\times m}$ is global Lipschitz. 
Also we assume 
that  each component function $f^{(j)}$  can be written of the form
\begin{equation}\label{eqn:AlternativeConstruction}
f^{(j)}(y) = a_j(y) y^{(j)} + b_j (y^{(-j)}), 
\end{equation}
where $a_j$ and	$b_{j}$ are two scalar 	functions in  $\R^d$ 
and $y^{(-j)} = \left( y^{(1)},\dots,y^{(j-1)},y^{(j+1)},\dots y^{(d)}\right)$. 
The chapter is organized as follows: In section 2 we give known results that are essential
for our purposes. In section 3, we construct the new explicit method and prove  
the always existence of a succession of the Linear Steklov  approximation as well as 
local Lipschitz conditions for its coefficients.  
In section 3 we prove the strong convergence of the LS method 
with one-half order using the Higham, Stuart and 
Mao (HSM) technique and in section 4 its  convergence rate is obtained. 
In section 5 we analyze numerically  the accuracy and efficiency of the proposed method 
applied to stochastic  differential equations with super-linear growth and locally Lipschitz coefficients. 
Finally we give some conclusions.









To assure existence and uniqueness of the solution of the SDE \eqref{eqn:SDE1}, we recall a 
classical result reported by \citeauthor*{Mao2013} in \cite{Mao2013}.  Also, we remind two results that establish bonds
for the moments of the solution see \cite{Higham2002b,Mao2007}.

	In this chapter we  study numerical approximations of vector It\^o stochastic differential equations (SDEs) with 
the form
\begin{equation}\label{eqn:SDE1}
	dy(t)
	=f(y(t))dt + g(y(t))dW(t), \quad 0\leq t\leq T,
	\quad y(0)=y_0.
\end{equation}
Here $(f^{(1)},\dots, f^{(d)}):\R^d \to \R^d$ and 
$g = (g^{(i,j)})_{i\in \{1,\dots,d\}, j\in\{1,\dots, m\}}:\R^d \to \R^{d\times m}$.
We will work with the standard setup, that is,  $y(t)\in \R^d$ for each $t$ and  $W(t)$ is a
$m$-dimensional standard Brownian motion on a filtered and complete probability space
$
	(
		\Omega ,\calF,(\calF_t)_{t\in[0,T]},\prob{}
	)
$,
with the filtration
$(\mathcal{F}_t)_{t\in[0,T]}$  generated by the Brownian process.
%
Also, we require the following assumptions over the coefficients 
	$f %= \left(f^{(1)},\dots, 
		%f^{(d)}\right)^{T}
	$,
	$
		g %=(g^{(1)},\ldots g^{(d)})^{T} $, $g^{(j)}:\R^d \to \R^d
	$.
\begin{hypothesis}\label[hypothesis]{ass:OSLC}
	The coefficients of SDE \eqref{eqn:SDE1} satisfy the following conditions:
	\begin{enumerate}[({H}-1)]
		\item \label{ass:C1Functions}
		The functions $f,g$ are in the class $C^{1}(\R^d)$.
		\item
		\textbf{Local, global Lipschitz condition}. For each integer $n$, there is a positive
		constant $L_{f}=L_{f}(n)$ such that
		$$
		|f(x)-f(y)|^2 %\vee |g(x)-g(y)|^2
		\leq L_{f}|x-y|^2 \qquad \forall x,y \in \R^d, \qquad |x|\vee|y|\leq n,
		$$
		and there is a positive constant $L_g$ such that
		$$
		|g(x)-g(y)|^2 \leq L_{g}|x-y|^2,
		\qquad  \forall x,y \in \R^d.
		$$ 
		\item\label{ass:MonotoneCondition}
		\textbf{Monotone condition.} There exist two positive constants $\alpha$ and $\beta$
		such that
		\begin{equation}\label{eqn:MonotoneCondition}
		\innerprod{x}{f(x)} +\frac{1}{2}|g(x)|^2
		\leq \alpha +\beta |x|^2, \qquad \forall x \in \R^d.
		\end{equation}
	\end{enumerate}
\end{hypothesis}
%
\begin{hypothesis}\label[hypothesis]{ass:ajBound} %\label{as:aiFunctions}
	Also, we require a special structure over the drift coefficient.
	\begin{enumerate}[({A}-1)]
		\item\label{ass:FunctionStructure}
		For each component function $f^{(j)}:\R^d:\to \R$, %$j \in \{1, \dots, d \}$,
		$j \in \{1,\dots, d\}$
		there are two locally Lipschitz functions $a_j:\mathbb{R}^{d} \to \mathbb{R}$, and
		$b_{j}:\mathbb{R}^{d-1} \to \mathbb{R}$ such that 
		\begin{equation}\label{eqn:AlternativeConstruction}
		f^{(j)}(y) = a_j (y) y^{(j)} + b_{j}(y^{(-j)}), \qquad
		y^{(-j)} = (y^{(1)}, \dots ,y^{(j-1)}, y^{(j+1)}, \dots, y^{(d)}).
		\end{equation}
		\item
		There is a positive constant $L_a$ such that
		\begin{equation}
		a_{j}(x) \leq L_a, \qquad \forall x\in \R^d, \quad j=1,\ldots, d.
		\end{equation}
		\item Each function $b_j(\cdot)$ satisfy the linear growth condition
		\begin{equation}\label{eqn:bjLinearGrowthCondition}
		|b_j(u^{(-j)})|^2 \leq L_{b}(1+|u|^2) , \qquad \forall x\in \R^d, \quad j=1,\ldots, d.
		\end{equation}
		\item
		Let $E_j:=\{x\in \R^d: a_j(x)=0\}$. Then, for each $x\in E_j$ there is an open ball of positive 
		radius and center $x$, $B_r(x)$, such that 
		$$
		\frac{\partial a_j(u)}{\partial u^{(j)}}
		\neq 0, \qquad \forall u \in B_r(x) \setminus E_j.
		$$
	\end{enumerate}
\end{hypothesis}
%-----------------------------------------------------------------------------------------------------------------------
%\todo{Addapt the proof to the multidimensional case}
%	The following lemma tell us that the functions $a_1^{(j)}$ in (H-\ref{ass:FunctionStructure})
%are upper bounded.
%\begin{lem}\label{lem:a1Bound}
%	Let \Cref{as:OSLC} holds. Then there is a positive constant $L_a$ such that
%	\begin{equation}
%		a_1^{(j)}(x)< L_a, \qquad \forall x \in \R, \quad j = {1, \ldots,d}.
%	\end{equation}
%\end{lem}
%\begin{pf}
%	First assume $|x|>1$. Then from the monotone condition \eqref{eqn:MonotoneCondition}
%	there is a positive constant $K\geq\max\{\alpha, \beta\}$ such that
%	$$
%	\innerprod{x}{a_1(x)x}\leq K
%	\left(
%		1+|x|^2
%	\right).
%	$$
%	Thus
%	$$
%		a_1(x)\leq K
%		\left(
%			\frac{1}{|x|^2}+1
%		\right)\leq 2K.
%	$$
%	Now, consider that $0\leq|x|\leq 1$. By the continuity of $a_1$ and since
%$[-1,1]$ is a
%	compact set, there is $\widetilde{x} \in [-1,1]$ such that
%	$$
%		a_1(x)\leq a_1(\widetilde{x}), \qquad \forall x\in[-1,1].
%	$$
%	Hence, taking $L_a:=\max\{2K,a_1(\widetilde{x})\}$ the upper bound is proved. \qed
%\end{pf}

In the next section, we will present the results for the existence and uniqueness of
the solution for the continuous problem \eqref{eqn:SDE1}.
%