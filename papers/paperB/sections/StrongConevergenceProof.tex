	Here, we state and prove the main result of this chapter, the strong convergence of the \SM method
 \crefrange{eqn:SSLSM1}{eqn:SSLSM2} for the solution of SDE \eqref{eqn:SDE1}.
The main idea of the proof consists in applying the technique discussed in \Cref{sec:HMS-Technique}.
We begin establishing the underlying convergence theorem.
\begin{thm} %[Strong Convergence of the \SM method]
%\begin{restatable}[Strong Convergence of the \SM]{theorem}{StrongConvergence}
	\label{thm:StrongConvergenceLSMethod}
	Let \Cref{ass:OSLC,ass:ajBound} hold, consider the \SM method \crefrange{eqn:SSLSM1}{eqn:SSLSM2} for the 
	SDE	\labelcref{eqn:SDE1}.
	Then there is a continuous-time extension $\overline{Y}(t)$ of the \SM solution $\{Y_k\}$ for which 
	$\overline{Y}(t_k)=Y_k$ and
	\begin{equation*}
	\lim_{h \to 0}
	\EX{
		\sup_{0\leq t \leq T}
		|\overline{Y}(t) - y(t)|^2	
	}=0.
	\end{equation*} 
\end{thm}
%\end{restatable}
%Before to apply the HMS technique we will prove
To proof this result, we initiate with the first step of the HMS technique, that is, we will show that the \SM method
for SDE \eqref{eqn:SDE1} is equivalent to the EM scheme applied to the conveniently modified SDE
	\begin{equation} \label{eqn:SDEMod}
		dy_h(t)= \varphi_{f_h}(y_h(t))dt +g_h(y_h(t))dW(t),
		\qquad y_h(0)=y_0,  \qquad t\in [0,T].
	\end{equation}
We formalize this as a Corollary of \Cref{lem:PhiFhProp}.
%======================================================================================================================
%                                                 STEP 1                                                              %
%======================================================================================================================

\begin{corollary}\label{col:SSSMeEMmod}
	Let \Cref{ass:OSLC,ass:ajBound} hold, then the \SM method for SDE \eqref{eqn:SDE1} is 
	equivalent to the EM scheme applied to the modified SDE \eqref{eqn:SDEMod}.
\end{corollary}
\begin{proof}
	Using the functions $\varphi_{f_h}(\cdot)$ and $g_h(\cdot)$ defined in \eqref{eqn:FunctionshDefinition} of 
	\Cref{lem:PhiFhProp}, we 
	can rewrite the \SM method \Crefrange{eqn:SSLSM1}{eqn:SSLSM2} as 
	$$
		Y_{k+1} = Y_k + h \varphi_{f_h}(Y_k) + g_h(Y_k)\Delta W_k,
	$$
	which is the EM approximation for the modified SDE \eqref{eqn:SDEMod}.
\end{proof}
%======================================================================================================================
%                                                 STEP 2                                                              %
%======================================================================================================================

	Now we proceed with the Step 2, that is,  we will prove that the solution  of the modified
SDE \eqref{eqn:SDEMod} has bounded moments and is close in uniform mean square norm to the solution of the SDE 
\eqref{eqn:SDE1}. In what follows we denote by $C$  a universal constant, that is, a positive constant
independent on h which value could change in occurrences.


\begin{lem}\label{lem:BoundAndConvergenceOfyh}
	Let \Cref{ass:OSLC,ass:ajBound,ass:HypThmSingularities} hold, then there is a universal 
	constant $C=C(p,T)>0$ and a sufficiently small
	step size $h$, such that for all $p>2$
	\begin{equation}\label{eqn:yh-MomentBounds}
		\m\left[
			\sup_{0\leq t \leq T}
				|y_h(t)|^p
		\right]
		\leq
			C
		\left( 
			1+\m |y_0|^p
		\right).
	\end{equation}
	Moreover
	\begin{equation}\label{eqn:yh-convergence}
	\lim_{h \to 0}
	\m\left[
	\sup_{0\leq t \leq T}
	|y(t)-y_h(t)|^2
	\right]=0.
	\end{equation}
\end{lem}
\begin{proof}
	By theorem \ref{thm:MaoCoercive} and inequality \eqref{eqn:h-MonotoneCondition}, 
	we have	 bound \eqref{eqn:yh-MomentBounds}.
	On the other hand, to prove \eqref{eqn:yh-convergence} we will use the properties of 
	$\varphi_{f_h}$ and the Higham's stopping time technique employed in \cite[Thm 2.2]{Higham2002b}. 
	Note that by relation \eqref{eqn:VarPhiEjc} of \Cref{lem:PhiFhProp} we have 
	\begin{align*}
		\varphi_{f_h}(x) 
			&= \Phi(h,a_j)(u) f^{(j)}(u) \1{E_j^c}(u) 
				+ f^{(j)}(u) \1{E_j}(u).
	\end{align*}
	
	By \Cref{ass:HypThmSingularities}  and since $f \in C^1(\R^d)$,   $\Phi(h,a_j)(\cdot)$ is bounded,
	hence, there is a positive constant $R_n$ which depends on $n$ such that 
	\begin{align*}	
		|\varphi_{f_h}^{(j)}(u) - f^{(j)}(u)|
		&\leq
			\1{E_j^c}(u)
			|f^{(j)}(u)|
			\left|
				\Phi(h,a_j)(u) - 1
			\right| \notag \\
		&\leq
			\1{E_j^c}(u)
			\left(
				L_{\Phi} + 1
			\right)
			|f(u)|	 \notag \\
		&\leq
		\1{E_j^c}(u) R_n(L_{\Phi}+1), \quad \forall u \in \R^d, \quad |u|\leq n,
	\end{align*}
	for each $j\in \{1,\dots, d\}$. 
	Moreover, we know by the proof of \Cref{lem:PhiFhProp} that
	\begin{equation*}
	 \lim_{
	 	\substack{
		 	h\to 0 \\
		 	u\in E_j^c	
	 	}
	 }
	 \Phi(h,a_j)(u) = 1.	 	
	\end{equation*}
	Also, we note that for each $j \in \{1, \dots , d\}$
	\begin{equation*}
	\lim_{h \to 0} F_h^{(j)}(u)
		=
		\lim_{h \to 0}
			e^{ha_j(u)} u^{(j)} + 
		\lim_{h \to 0}
			\left(
				\frac{e^{ha_j(u)}-1}{a_j(u)}
				\1{E_j^c}(u)
				+h \1{E_j}(u)
			\right)
			b_j(u^{(j)}) 
		= u^{(j)},
	\end{equation*}
	hence
	$%\begin{equation*}
		\displaystyle
		\lim_{h\to 0} F_h(u)=u.
	$ %\end{equation*}
	Consequently, given $n>0$ there is  a function $K_n(\cdot):(0,\infty)\to (0,\infty)$, such that
	$K_n(h)\to 0$ when $h \to 0$ and
	\begin{equation}\label{eqn:PhihGhKRhBound}
		|\varphi_{f_h}(u)-f(u)|^2 \vee |g_h(u)-g(u)|^2
		\leq K_n(h) \qquad \forall u\in \R^d, \quad |u| \leq n.
	\end{equation}
	Now, using that both $f$, $g$ are $C^{1}$, there is  a constant $H_n>0$ such that
	\begin{equation}\label{eqn:f-gHRBound}
		|f(u)-f(v)|^2 \vee |g(u)-g(v)|^2
		\leq H_n |u-v|^2\qquad \forall u,v \in \R^d, |u|\vee |v| \leq n.
	\end{equation}
	
		On the other hand, by \Cref{lem:MomentBound} and inequality \eqref{eqn:yh-MomentBounds} we obtain
	\begin{equation*}
		\m\left[
			\sup_{0\leq t \leq T}
				|y(t)|^p
		\right]
		\vee
		\m\left[
			\sup_{0\leq t \leq T}
				|y_h(t)|^p
		\right]
		\leq
		K := C
		\left( 
			1+\m |y_0|^p
		\right).
	\end{equation*}
	Now, we define the stopping times
	\begin{equation}\label{eqn:StoppingTimes}
		\tau_n := 
			\inf\{
				t\geq 0: |y(t)|\geq n
			\},
		\qquad
		\rho_n := 
			\inf\{
				t\geq 0: |y_h(t)|\geq n
			\},
		\qquad
		\theta_n:=
			\tau_n \wedge \rho_n,
	\end{equation}
	and the difference function
	\begin{equation*}
		e_h(t):= y(t) - y_h(t).
	\end{equation*}
	From the Young's inequality \eqref{eqn:YoungsInequality}, we deduce that for any $\delta>0$ 
	\begin{align}
		\m\left[
			\sup_{0\leq t\leq T}
			|e_h(t)|^2
		\right]
		&=
			\m\left[
				\sup_{0\leq t\leq T}
				|e_h(t)|^2
				\1{\tau_n>T,\rho_n>T}
			\right]
			+
			\EX{
				\sup_{0\leq t\leq T}
				|e_h(t)|^2
				\1{\tau_n \leq T \text{ or } \rho_n \leq T}
			}\notag\\
		&\leq
			\EX{
				\sup_{0\leq t\leq T}
				|e_h(t\wedge \theta_n)|^2
				\1{\theta_n \geq T}
			}
			+\frac{2\delta}{p}
			\EX{
				\sup_{0\leq t\leq T}
				|e_h(t)|^p 
			}\notag \\
		&+
			\frac{1-2/p}{\delta^{2/(p-2)}}
			\Prob{\tau_n \leq T \text{ or } \rho_n \leq T}.
	\label{eqn:AfterYoungIneq}
	\end{align}
	We proceed to bound each term on the right-hand side of inequality \eqref{eqn:AfterYoungIneq}.
	By \Cref{lem:MomentBound}, $y(t)$ has bounded moments, hence 
	there is a positive constant $A$ such that
	\begin{equation}\label{eqn:BoundProbTauR}
		\Prob{\tau_n\leq T}
		=
			\EX{\1{\tau_n<T}\frac{|y(\tau_n)|^p}{n^p}}
		\leq
			\frac{1}{n^p}\EX{\sup_{0\leq t\leq T}|y(t)|^p} \leq \frac{A}{n^p},
			\qquad \text{for } p\geq 2.
	\end{equation}
	The same conclusion can be drawn for $\rho_n$, then
	\begin{equation} \label{eqn:BoundProbTauRorRhoR}
		\Prob{\tau_n \leq T \text{ or } \rho_n \leq T}
		\leq
			\Prob{\tau_n\leq T}+\Prob{\rho_n\leq T}
		\leq
		\frac{2A}{n^p}.
	\end{equation}
	Now, using inequality \eqref{eqn:SingleHolder} and \Cref{lem:MomentBound} we have
	\begin{equation} \label{eqn:ehMomentBound}
		\EX{
			\sup_{0\leq t \leq T}
			|e_h(t)|^p
		}
		\leq
		2^{p-1}
		\EX{
			\sup_{0 \leq t \leq T}
			\left(
			|y(t)|^p + |y_h(t)|^p
			\right)
		}
		\leq 2^pA.
	\end{equation}
%
	So, combining  bound \eqref{eqn:BoundProbTauRorRhoR} with \eqref{eqn:ehMomentBound}
	in inequality \eqref{eqn:AfterYoungIneq} we obtain
	\begin{align}
		\EX{
			\sup_{0\leq t \leq T}
			|e_h(t)|^2
		}
		&\leq
			\EX{
				\sup_{0\leq t\leq T}
				|e_h(t\wedge \theta_n)|^2
				\1{\theta_n \geq T}
			}
	%\notag \\
			+\frac{2^{p+1}\delta A}{p}
			+\frac{2(p-2)A}{p\delta^{2/(p-2)}n^p}. \label{eqn:TermToBound}
	\end{align}
	Next, we show that the first term of \eqref{eqn:TermToBound} is bounded. Adding conveniently terms yields
	\begin{align*}
		e_h(t\wedge\theta_n) 
			&=
			\int_{0}^{t\wedge\theta_n}
			\left[
				f(y(s)) - f(y_h(s))+f(y_h(s))
				-\varphi_{f_h}(y_h(s))
			\right]ds \notag \\
			&+
			\int_{0}^{t\wedge\theta_n}
			\left[
				g(y(s)) - g(y_h(s))+g(y_h(s))
				-g_h(y_h(s))
			\right]dW(s).
	\end{align*}
	Using bounds \eqref{eqn:PhihGhKRhBound} and \eqref{eqn:f-gHRBound}, the Cauchy-Schwarz, and
	Doob martingale inequalities, we get
	\begin{align*}
		\EX{\sup_{0\leq t \leq \tau}|e_h(t\wedge\theta_n)|^2}
		&\leq 
		4H_n(T+4)
		\int_{0}^{\tau}
			\EX{\sup_{0\leq t \leq \tau}|e_h(t\wedge\theta_n)|^2} ds 
		+
		4T(T+4)K_n(h).\notag
	\end{align*}
	The Gronwall inequality now yields
	\begin{align*}
		\EX{\sup_{0\leq t \leq T}|e_h(t\wedge\theta_R)|^2}
		&\leq
			4T(T+4)K_n(h)\exp(4H_n(T+4)T). %\notag \\
	%+
	%\frac{2^{p+1}\delta A}{p}
	%+
	%\frac{(p-2)2A}{p\delta^{2/(p-2)}R^p}.
	\end{align*}
	Hence, given $\epsilon>0$ for any $\delta>0$ such that
	$
		2^{p+1}\delta A/p< \epsilon/3,
	$
		we can take $n>0$ verifying
	$
		(p-2)2A/(p\delta^{2/(p-2)}n^p)<\epsilon/3.
	$
	Moreover, we can take $h$ sufficiently small such that
	$
		4T(T+4)K_n(h)e^{4H_n(T+4)T} < \epsilon/3. 
	$ 
	It follows immediately that
	$$
		\EX{\sup_{0\leq t \leq T}|e_h(t)|^2}
		<
			\epsilon/3
			+\epsilon/3
			+\epsilon/3
		=\epsilon,
	$$ which is the desired conclusion.
\end{proof}
%======================================================================================================================
%                                                 STEP 3                                                              %
%======================================================================================================================

	Next, we proceed with Step 3, in which we establish that \SM method has bounded moments.
\begin{lem}\label{lem:SSSMMomentBounds}
	Let \Cref{ass:OSLC,ass:ajBound,ass:HypThmSingularities} hold. Then for each $p\geq 2$ there is 
	a universal positive constant  $C=C(p,T)$ 
	such that the explicit \SM method
	\begin{equation*}
		\m\left[
		\sup_{kh \in [0,T]}
		|Y_k|^{2p}
		\right]\leq C.
	\end{equation*}
\end{lem}
\begin{proof}
Denoting by  $A^{(i)}_k:= A^{(i)} (h,Y_k)$ for $i=1,2$ and $b_k:=b(Y_k)$, we use
a split formulation of the \SM scheme \crefrange{eqn:SSLSM1}{eqn:SSLSM2} as follows:
\begin{eqnarray*}
	Y_{k}^{{\star}^{(j)}} &=& A^{(1)}_k Y_k + A^{(2)}_k b_k, \label{split1}\\
	Y_{k+1}^{(j)}&=& Y_k^{{\star}^{(j)}} + g^{(j)}(Y_k^{\star})\, \Delta W_k 
	\label{split2},
\end{eqnarray*}
from the first step of this split scheme, using (A-3) and 
the Cauchy-Schwartz inequality, we get
\begin{eqnarray}
|Y_k^{\star}|^{2}
&\leq&
|A^{(1)}_k |^2 |Y_k|^2  
+ 2 \innerprod{A^{(1)}_k Y_k}{A^{(2)}_k Y_k 
	b_k}
+|A^{(2)}_k|^2 |b_k|^2\nonumber\\
&\leq&
|A^{(1)}_k|^2 |Y_k|^2  
+ 2 \sqrt{L_b} d|A^{(1)}_k||A^{(2)}_k||Y_k|(1+|Y_k|)
+L_b|A^{(2)}_k|^2 (1+|(Y_k)|^2).\label{leqn:Yn2Bound} 
\end{eqnarray}
From (A-2), we can deduce that
\begin{dmath}[label=eqn:A1Bound]
	|A^{(1)}_k|^2 
	=
	\left|
	\diag
	\left(
	e^{ha_1(Y_k)}, \dots, e^{ha_d(Y_k)} 
	\right)
	\right|^2
	\leq L_{A^{(1)}},		
\end{dmath}
where $L_{A^{(1)}}=d\, e^{ 2 T L_a}$ and also by \eqref{eqn:PhiBound}, we can derive that
\begin{align}
	|A^{(2)}(h,Y_k)|^2 
	&=
	\left|
	h 
	\diag
	\left(
	\1{E_1}(Y_k)
	+\1{E_1^c}(Y_k)\Phi_1(Y_k), 
	\dots,
	\1{E_d}(Y_k)
	+\1{E_d^c}(Y_k) \Phi_d(Y_k)
	\right)
	\right|^2 \notag \\
	%
	&\leq
	\sum_{j=1}^{d}
	\left(
	\1{E_j^c}
	|h\Phi_j(Y_k)|^2
	+ h^2
	\right)
	\leq
	2 e^{2 L_a  T}
	\sum_{j=1}^d
	\frac{1}{a_j^*} + d T^2\leq L_{A^{(2)}}.
	\label{eqn:A2Bound}
\end{align}
Substituting \eqref{eqn:A1Bound} and \eqref{eqn:A2Bound}  on  inequality \eqref{leqn:Yn2Bound} yields
\begin{eqnarray*}\label{eqn:YkStarBound}
	|Y_k^{\star}|^2
	&\leq&	L_{A^{(1)}} |Y_k|^2
	+ 2 d \sqrt{L_{A^{(1)}} L_{A^{(2)}} L_b }\,|Y_k|(1+|Y_k|)
	+L_{A^{(2)}} L_b (1+|Y_k|^2)		
	\leq C(1+|Y_k|^2),
\end{eqnarray*}
where $C\geq L_{A^{(1)}}+ 2 d \sqrt{L_{A^{(1)}} L_{A^{(2)}} L_b} + 
L_{A^{(2)}} L_b$. Applying  bound \eqref{eqn:YkStarBound} 
in the  second step of the split scheme, we get
\begin{equation*}
	|Y_{k+1}|^2
	\leq
	C \left(
	|Y_k|^2 + 1
	\right)
	+ 2\innerprod{Y^{\star}_k}{g(Y^{\star}_k) \Delta W_k}
	+ \left|g(Y^{\star}_k) \Delta W_k \right|^2.
\end{equation*}
Now, we choose two integers $N,M$ such that $Nh\leq Mh \leq T$. So, adding 
backwards we obtain
\begin{equation*}
	|Y_N|^2
	\leq
	S_N\left(
	\sum_{j=0}^{N-1}
	(1+|Y_j|^2)
	+
	2\sum_{j=0}^{N-1}
	\innerprod{Y_j^{\star}}{g(Y_j^{\star}) \Delta W_j}
	+
	\sum_{j=0}^{N-1}
	\left|
	g(Y_j^{\star}) \Delta W_j
	\right|^2
	\right),
\end{equation*}
where	$S_N:=	\sum_{j=0}^{N-1}C^{N-j}$. Raising both sides to the 
power $p$,  we get
\begin{align}\label{eqn:RelationToBound}
	|Y_N|^{2p}	
	&\leq
	6^{p} S_N^p
	\left(
	N^{p-1}
	\sum_{j=0}^{N-1}
	(1+|Y_j|^{2p})	
	+
	\left|
	\sum_{j=0}^{N-1}
	\innerprod{Y_j^{\star}}{g(Y_j^{\star}) \Delta W_j}
	\right|^p
	+
	N^{p-1}
	\sum_{j=0}^{N-1}
	\left|
	g(Y_j^{\star}) \Delta W_j
	\right|^{2 p}				
	\right).
\end{align}
Now we will show that the second and third terms of inequality \eqref{eqn:RelationToBound} are bounded.
We denote by $C=C(p,T)$ a generic positive constant which does not depend on  the step size $h$ and whose
value may change between occurrences.	Next, applying the Bunkholder-Davis-Gundy
inequality  \cite{Mao2007}, we have
\begin{eqnarray}\label{eqn:BoundSecondTerm}
\m
\left[
\sup_{0\leq N \leq M}
\left|
%\exp(2hpNL)
\sum_{j=0}^{N-1}
\innerprod{Y_j^{\star}}{g(Y_j^{\star})\Delta W_j}
\right|^{p}
\right]
&\leq&
C\m
\left[
\sum_{j=0}^{N-1}
|Y_j^{\star}|^2
|g(Y_j^{\star})|^2
h
\right]^{p/2}
\notag\\
&\leq&
C h^{p/2}M^{p/2-1}
\m
\sum_{j=0}^{M-1}
|Y_j^{\star}|^p (\alpha +\beta |Y_j^{\star}|^2)^{p/2}
\notag\\
&\leq&
2^{p/2-1}C T^{p/2-1} h  
\m
\sum_{j=0}^{M-1}
(\alpha^{p/2}|Y_j^{\star}|^p +\beta^{p/2} |Y_j^{\star}|^{2p})
\notag\\
&\leq&
C h
\m
\sum_{j=0}^{M-1}
(1+2|Y_j^{\star}|^p + |Y_j^{\star}|^{2p})
\notag\\
&\leq&
C 
+ 
C h 
\sum_{j=0}^{M-1}
\m|Y_j|^{2p},				
\end{eqnarray}
Now, using the Cauchy-Schwartz inequality, the monotone condition 
\eqref{eqn:MonotoneCondition} and bound \eqref{eqn:YkStarBound}, we obtain
\begin{eqnarray}
\m\left[
\sup_{0\leq N \leq M}
\sum_{j=0}^{N-1}
\left|
g(Y_j^{\star})\Delta W_j
\right|^{2p}	
\right]
&\leq&	
\sum_{j=0}^{M-1}
\m
\left|
g(Y_j^{\star})
\right|^{2p}
\m
\left|
\Delta W_j
\right|^{2p}
\notag \\
&\leq&
C h^p
\sum_{j=0}^{M-1}
\m
\left[
\alpha +\beta|Y^{\star}_j|^2
\right]^p
\notag\\
%
&\leq&
Ch^p
\sum_{j=0}^{M-1}
\m
\left[
\alpha ^p +\beta^p |Y^{\star}_j|^{2p}
\right]
\notag\\
&\leq&
Ch^{p-1}
+
Ch^p \sum_{j=0}^{M-1}
\m|Y_j|^{2p} \label{eqn:BoundThirdTerm}.
\end{eqnarray}
Thus, combining bounds \eqref{eqn:BoundSecondTerm} and \eqref{eqn:BoundThirdTerm} with inequality 
\eqref{eqn:RelationToBound}, we can assert that
\begin{align}
	\EX{
		\sup_{0\leq N \leq M}
		|Y_N|^{2p} 
	}
	% 		\leq
	% 			C(M,T) + C(M,T)(1+h)
	% 			\sum_{j=0}^{M-1}
	% 				\m|Y_j|^{2p}
	\leq	
	C +C(1+h) 
	\sum_{j=0}^{M-1}
	\EX{
		\sup_{0\leq N \leq j}
		|Y_N|^{2p}
	}	
	.
\end{align}
Finally, using the discrete-type Gronwall inequality \cite{Mao2007}, we conclude that
\begin{align*}
	\EX{
		\sup_{0\leq N \leq M}
		|Y_N|^{2p} 
	}	
	&\leq
	C e^{C(1+h)M} 
	\leq 
	C e^{C(1+T)}<C,
\end{align*}
since the constant C does not depend on $h$, the proof is complete.
\end{proof}

%======================================================================================================================
%                                                 STEP 4                                                              %
%======================================================================================================================
	
	Since the \SM scheme has bounded moments, we now proceed whit Step 4, that is,  we will obtain a
continuous extension of the \SM method with bounded moments. 
Let $\{Y_k\}$ denote the \SM solution of SDE \eqref{eqn:SDE1}.
By \Cref{col:SSSMeEMmod}, we conveniently made a continuous extension for the \SM approximation, from the 
time continuous extension of the EM method \eqref{eqn:EMContinuousExtension}.
Also, we prove that the  moments of the Linear Steklov extension remains bounded.
\begin{corollary}\label{col:ContinuousExtBoundedMoments}
	Let \Cref{ass:OSLC,ass:ajBound,ass:HypThmSingularities} hold and suppose  $0<h<1$ and $p\geq 
	2$. Then there is a continuous extension $\overline{Y}(t)$ of $\{Y_k\}$  and a universal constant $C=C(T,p)$ such 
	that
	\begin{equation*}
		\EX{\sup_{0\leq t \leq T} |\overline{Y}(t)|^{2p} }
		\leq C.
	\end{equation*}
\end{corollary}
	\begin{proof}
		We take $t=s+t_k$ in $ [0,T]$, $\Delta W_k(s):= W(t_k+s)- W(t_k)$ and $0\leq s <h$.
		Then we define 
		\begin{equation}\label{eqn:SSLSContinuousExtension}
			\overline{Y}(t_k+s):= Y_k + s \varphi_{f_h}(Y_k) + g_h(Y_k)\Delta W_k(s),
		\end{equation}
		as a continuous extension of the \SM scheme. We proceed to show that $\overline{Y}(t)$ has bounded moments.
		By \Cref{lem:PhiFhProp}, we have $Y_k^{\star}= Y_k + h \varphi_{f_h}(Y_k)$. 
		Then for $\gamma = s/h$, it follows that
		\begin{align*}
			Y_k + s \varphi_{f_h}(Y_k)
			&= 
			\gamma (Y_k + h \varphi_{f_h}(Y_k)) +(1-\gamma)Y_k\\
			&=
			\gamma Y_k^{\star} + (1-\gamma)Y_k.
		\end{align*}
		Hence, we can rewrite the continuous extension \eqref{eqn:SSLSContinuousExtension} as
		\begin{align}
			\overline{Y}(t) &=
			\gamma Y^{\star}_k + (1-\gamma) Y_k +g_h(Y_k) \Delta W_k(s). %\label{eqn:SSLSMConExt}
			\notag
		%t&=t_k+s,  \qquad  \gamma = s/h\qquad s\in {[0,h)]}.\notag
		\end{align}
		%
		Combining this relation with  the inequalities \eqref{eqn:YkStarBound} and \eqref{eqn:SingleHolder}, we get
		\begin{align}
			|\overline{Y}(t_k+s) |^2 
			&\leq
				3\left[
					\gamma C
					+
					\left(
						\gamma C +1 - \gamma
					\right)
					|Y_k|^2
					+
					|g_h(Y_k)\Delta W_k(s)|^2
			\right] \notag\\
		&\leq
			C
			+
			C
			\left(
				|Y_k|^2 + |g_h(Y_k)\Delta W_k(s)|^2
			\right).
		\notag
		\end{align}
	Thus, 
	\begin{align}
		\sup_{0\leq t\leq T} |\overline{Y}(t)|^{2p}
		&\leq
			\sup_{0\leq kh\leq T}
			\left[
				\sup_{0\leq s\leq h}
					|\overline{Y}(t_k+s)|^{2p} 
			\right] 
		\notag\\
		&\leq
			\sup_{0\leq kh\leq T} 
			\left[
				\sup_{0\leq s\leq h}
					C 
					\left(
						1 + |Y_k|^{2p} + |g_h(Y_k)\Delta W_k(s)|^{2p}
					\right)
			\right],
		\label{eqn:BeforeDoob}
	\end{align}
	for $t\in [0,T]$.
	%
	Now taking a non negative integer $0 \leq k \leq N$ such that $0\leq Nh \leq T$. From the bond 
	\eqref{eqn:BeforeDoob}, we get
	\begin{align}
		\sup_{0\leq t\leq T} |\overline{Y}(t)|^{2p}
		&\leq 
			C
			\left(
				1
				+
				\sup_{0\leq kh\leq T} 
					|Y_k|^{2p}
					+
					\sup_{0\leq s\leq h}
						\sum_{j=0}^N
							|g_h(Y_j)\Delta W_j(s)|^{2p}
			\right) \label{eqn:SumDiffusion}.
	\end{align}
	So, using the Doob's Martingale inequality \eqref{eqn:DoobMartingaleInequality},
	\Cref{lem:SSSMMomentBounds} and that $g_h$ is a locally 
	Lipschitz function, we can bound each term of the inequality \eqref{eqn:SumDiffusion},  as follows
	\begin{align}
		\EX{
			\sup_{0 \leq s \leq h} |g(Y_j) \Delta W_j(s)|^{2p}
		}
		&\leq
			\left(
				\frac{2p}{2p-1}
			\right)^{2p}
			\m|g_h(Y_j)\Delta W_j(h)|^{2p}
			\notag
			\\
		&\leq
			C \m |g_h(Y_j)|^{2p} \m |\Delta W_j(h)|^{2p} \notag\\
		&\leq
			C h^p
			\left(
				1 + \m|Y_j|^{2p}
			\right) \notag \\
		& \leq C h, \label{eqn:BeforeConclusion}
	\end{align}
	for each $j \in \{0,\dots, N\}$.
	Since $Nh\leq T$, combining the bounds \eqref{eqn:SumDiffusion} and \eqref{eqn:BeforeConclusion} we 
	get the desired conclusion.
	\end{proof}
%======================================================================================================================
%                                                 STEP 5                                                              %
%======================================================================================================================

	Once we have carried out all the previous steps, we can prove the \Cref{thm:StrongConvergenceLSMethod} by Step 5.%
%We are now in a position to execute the \textbf{Step 5}.
\begin{proof}[Proof of \Cref{thm:StrongConvergenceLSMethod}]
	First,  note that by inequality \eqref{eqn:SingleHolder}, we have
	\begin{align}\label{eqn:AfterTriangle}
		\EX{\sup_{0\leq t \leq T}|\overline{Y}(t) - y(t)|^2}
		&\leq
		2\EX{
			\sup_{0\leq t \leq T}
			|\overline{Y}(t) - y_h(t)|^2
		}
		+
		2\EX{
			\sup_{0\leq t \leq T}
			|y_h(t) - y(t)|^2
		}.
	\end{align}
	Using \Cref{lem:BoundAndConvergenceOfyh}, which was established in the Step 2, yields
	\begin{equation}\label{eqn:SeconTermZeroLim}
		\lim_{h\to 0}
		\EX{
			\sup_{0\leq t \leq T}
			|y_h(t) - y(t)|^2
		} = 0.
	\end{equation}
	
		It remains to prove that the first term of the right hand side in inequality \eqref{eqn:AfterTriangle} 
		decreases to zero
	when $h$ tends to zero. Recalling that:
	\begin{enumerate}[i)]
	%\begin{inparaenum}[\itshape i\upshape)]
		\item 
			By \Cref{lem:BoundAndConvergenceOfyh}, the solution of the modified SDE \eqref{eqn:SDEMod}, $y_h$, has
			$p$-bounded moments ($p\geq 2$).
		\item
			By \Cref{col:ContinuousExtBoundedMoments}, the \SM continuous extension for the SDE \eqref{eqn:SDE1},
			$\overline{Y}(t)$, has bounded moments and it is equivalent to the EM extension for the modified SDE 
			\eqref{eqn:SDEMod}.
		%\end{inparaenum}
	\end{enumerate}
	Hence, we can apply 
	%\cite[Thm. 2.2]{Higham2002b} 
	\Cref{thm:HighamMaoStuart} to conclude that
	\begin{equation}\label{eqn:FirstTermZeroLim}
		\lim_{h\to 0}
		\EX{
			\sup_{0\leq t \leq T}
			|\overline{Y}(t) - y_h(t)|^2
		} = 0.
	\end{equation}
	Finally, combining the limits \eqref{eqn:SeconTermZeroLim} and \eqref{eqn:FirstTermZeroLim} with 
	inequality \eqref{eqn:AfterTriangle} gives
	\begin{align*}
		\lim_{h \to 0}
		\EX{
			\sup_{0\leq t \leq T}
			|\overline{Y}(t) - y(t)|^2
		}
		&\leq	
		2\lim_{h\to 0}
		\EX{
			\sup_{0\leq t \leq T}
			|\overline{Y}(t) - y_h(t)|^2
		}
		\\
		&
		+
		2\lim_{h\to 0}
		\EX{
			\sup_{0\leq t \leq T}
			|y_h(t) - y(t)|^2
		} = 0,
	\end{align*}
	which proves the theorem. 
\end{proof}
%