\section{Conclusions}
	We have constructed a new way  to design numerical methods for SDEs based on the Steklov average.
First we presented a scalar scheme  originated in an exact discretization for the deterministic version of the SDEs 
with desired stability properties --- the Steklov method.
We verified its convergence and stability over a standard globally Lipschitz setup and compared its performance with a 
competitive solvers.
%
Also, we have extended the explicit Steklov scheme for vector SDE by developing a new version based on a linearized 
Steklov average. This method is constructed on the basis that the drift function can be rewritten in the linearized 
form. Moreover, strong order one-half convergence has been proved for our explicit linear method and we have presented 
several applications formulated with the LS scheme. Finally, high-performance of the Linear Steklov method have been 
analyzed in diverse problems, even for SDEs with super-linear diffusion.
Future work will be focused on the following problems:
%
\begin{itemize}
	\item
		Numerical evidence suggests that the Steklov methods are suitable for SDEs with super-	
		linear growth diffusion. 
		So, one should prove this claim.
	\item
		Since we have proved strong convergence, we would to apply the Multilevel Monte Carlo 
		approach to
		the Brownian Dynamics Simulation using Steklov type schemes.		
	\item
		The schemes presented here have a simple structure like the	Euler-Maruyama family, so it 
		is possible	to formulate versions of the Tamed, Milstein, Balanced, Theta, 
		Runge-Kutta methods by approximating the drift term by its Steklov average.  
	\item	
		Also, it is viable to study stability of the LS method usibg the theory of random 
		dynamical systems.
	\item
		Furthermore, a natural extension of this work would be to design Steklov type schemes 
		for more general SDEs, that is, SDEs with delay, Poisson jumps or partial derivatives. 	
\end{itemize}

 
 
