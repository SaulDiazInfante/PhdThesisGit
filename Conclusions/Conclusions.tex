\section{Conclusions}
	We have been proposed a new way  to design numerical methods for SDEs based on the Steklov average.
Two Steklov type methods were constructed to evidence our new approach.
First we presented a scalar scheme with good stability properties ---the Steklov method.
We verify its convergence and stability over a standard globally Lipschitz setup and compare its performance with a 
competitive solvers.
%
With the Linear Steklov scheme we obtain a extension over multidimensional and locally Lipschitz 
context. Also we proved its one-half convergence order and evidence its accuracy by simulation, even for 
SDEs with super-linear diffusions. However, we see a world of problems that should be consider. We mention  just
a few of them.

\begin{itemize}
	\item
		The numerical evidence suggest that the scalar Steklov and the LS methods works under diffusions with 
		super-linear growth. So, one of the possible future directions points to prove this claim. 
	\item
		Since we have been prove strong convergence, we would to apply the Multilevel Monte Carlo approach to
		the Brownian Dynamics Simulation using Steklov type schemes.		

	\item
		The schemes presented here follow structure of the 
		Euler-Maruyama family, so we see that is possible formulate schemes of type Tamed, Milstein, Balanced, Theta, 
		Runge-Kutta  and other kinds of methods, by replacing evaluations of the drift $f$ by its Steklov average.
	\item	
		Also, we see viable to study stability of the LS method trough the theory of random dynamical systems.
	\item
		Furthermore, a natural extension would be to design Steklov type schemes for more general SDEs, that is, SDEs 
		with delay, Poisson jumps or partial derivatives. 	
\end{itemize}


%

 
 
