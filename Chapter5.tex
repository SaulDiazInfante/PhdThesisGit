\documentclass[1p, sort&compress, preprint, number]{elsarticle}
\usepackage[english]{babel}
\usepackage{amsmath,amssymb}
\usepackage{breqn}
\usepackage{amsthm}
\usepackage{thmtools}
\usepackage{bbold}
\usepackage{booktabs}
\usepackage{graphicx}
\usepackage[utf8]{inputenc}
\usepackage[T1]{fontenc}
\usepackage{caption}
\usepackage{siunitx}
\usepackage{hyperref}
\usepackage[labelfont=bf,textfont={sl,bf},lofdepth,lotdepth]{subfig}
\usepackage{xspace}
\usepackage{color}
\usepackage{proba}
\usepackage[usenames,dvipsnames,svgnames,table]{xcolor}
\usepackage{cleveref}
\usepackage{paralist}
\usepackage{todonotes}
\usepackage{bigints}
\usepackage[title,titletoc,toc]{appendix}
\usepackage{import}
\usepackage{natbib}
\usepackage{siunitx}
%%%%%%%%%%%%%%%%%%%%%%%%%%%%%%%%%%%%%%%%%%%%%%%%%%%%%%%%%%%%%%%%%%%%%%%%%%%%%%%%%%%%%%%%%%
%                                  MODIFICACIONES                                        %
%%%%%%%%%%%%%%%%%%%%%%%%%%%%%%%%%%%%%%%%%%%%%%%%%%%%%%%%%%%%%%%%%%%%%%%%%%%%%%%%%%%%%%%%%%
\oddsidemargin=0.5cm
\textwidth=15.5cm
%%%%%%%%%%%%%%%%%%%%%%%%%%%%%%%%%%%%%%%%%%%%%%%%%%%%%%%%%%%%%%%%%%%%%%%%%%%%%%%%%%%%%%%%%%
%                                 FORMATOS                                               %
%%%%%%%%%%%%%%%%%%%%%%%%%%%%%%%%%%%%%%%%%%%%%%%%%%%%%%%%%%%%%%%%%%%%%%%%%%%%%%%%%%%%%%%%%%
\DeclareMathAlphabet{\mathpzc}{OT1}{pzc}{m}{it}
\newtheorem{dfn}{Definition}
\newtheorem{thm}{Theorem}[section]
\newtheorem{pro}{Proposition}[section]
\newtheorem{lem}{Lemma}[section]
\newtheorem{definition}{Definition}[section]
\newtheorem{corollary}{Corollary}[section]
\newtheorem{consequence}{CONSEQUENCE}[section]
\newtheorem{remark}{Remark}[section]
\newtheorem{example}{\bf Example}[section]
%\newtheorem{proof}{\bf Proof}[section]
\newtheorem{assumption}{Assumption}[section]
\newproof{pf}{Proof}
\newproof{Proof}{Proof}
%declaration theorems for appendix
\declaretheorem[numbered=no, name=H\"{o}lder]{Holder}
\declaretheorem[numbered=no, name=Young]{Young}
\declaretheorem[numbered=no, name=Minkowski]{Minkowski}
\declaretheorem[numbered=no, name=Doob's Martingale Inequality]{Doobs}
\declaretheorem[numbered=no, name=Burkholder–Davis–Gundy inequality]{bdg}
\declaretheorem[numbered=no, name=Gronwall inequality]{Gronwall}
\declaretheorem[numbered=no, name=Discrete Gronwall Inequality]{DiscreteGronwall}
\declaretheorem[numbered=no, name=A usefull inequality]{Usefull}
%
\providecommand*{\lemautorefname}{Lemma}
\providecommand*{\thmautorefname}{Theorem}
\providecommand*{\assumptionautorefname}{Assumption}

\newcommand{\normL}[1]{\left[\mathbb{E}\left|#1\right|^2\right]^{1/2}}
\newcommand{\ms}[1]{\mathbb{E}\left|#1\right|^2}
\newcommand{\mep}[1]{\mathbb{E}|#1|^p}
\newcommand{\m}[1]{\mathbb{E}#1}
\newcommand{\Prob}[1]{\mathbb{P}\left[#1\right]}
\newcommand{\meanp}[2]{\mathbb{E}\left|#1\right|^{#2}}
\newcommand{\condexp}[2]{\mathbb{E}\left[#1|#2\right]}
\newcommand{\lftrght}[3]{\left#2 #1\right #3}\DeclareMathOperator{\tr}{tr}
\newcommand{\innerprod}[2]{\left\langle#1, #2\right\rangle}
\newcommand*{\eg}{e.g.,\xspace}
\newcommand*{\ie}{i.e.,\xspace}
\newcommand{\SM}{LS\xspace}
\newcommand{\crefrangeconjunction}{--}
\crefrangeformat{equation}{(#3#1#4)--(#5#2#6)}
\DeclareMathOperator{\diag}{diag}
\DeclareMathOperator*{\as}{a.s.}
\DeclareMathOperator{\Tr}{Tr}

%opening
\begin{document}
	\begin{frontmatter}
		\title{
			Numerical Experiments with the Explicit Linear Steklov Method.
		\tnoteref{t1}
		}%,t2}}
		\tnotetext[t1]{
			This work has been partially
			supported by CONACYT project *****
		}
		\author[sj]{S. D\'{\i}az-Infante}
		\ead{sauld@cimat.mx}
		\author[sj]{S. Jerez}
		\ead{jerez@cimat.mx}
		\address[sj]{Split Step Linear Steklov Method 
		Department of Applied Mathematics, CIMAT, Guanajuato, Gto., Mexico,
		36240.
		}
		\begin{abstract}
			In this document, we develop a set of examples where the Steklov Schemes can be applied
			with good results. The examples was taken from common and variated application areas.
		\end{abstract}
		\begin{keyword}
			stochastic differential equations, Steklov average,
			explicit methods, convergence, asymptotic stability,
		\end{keyword}
	\end{frontmatter}
		\tableofcontents
		\pagebreak
%
\todo{Write a general introduction}
	\section{Population  Dynamics}
		\subsection{Stochastic Lotka-Volterra model }
			\import{./papers/paperB/sections/StronConvergenceSSLSM/}{LotkaVolterraArnoldSDE}
		\subsection{A Stochastic Model for Internal HIV Dynamics}
			\import{./papers/paperB/sections/StronConvergenceSSLSM/}{StochasticAIDSHIV}
		\subsection{Stochastic Bone Remodeling}
	\section{Nonlinear Stochastic Oscillators}
		\import{./papers/paperB/sections/StronConvergenceSSLSM/}{IntroOscilators}
		\subsection{Stochastic Duffing Van Der Pol Equation}
			\import{./papers/paperB/sections/StronConvergenceSSLSM/}{StochasticDuffingVanDerPolSchenck}
		%\subsection{Simplified Duffing Van Der Pol Equation}
%			\import{./papers/paperB/sections/StronConvergenceSSLSM/}{SimplifiedVanDerPol}
%		\subsection{Generalized Van der Pol Oscillator with multiplicative noise}
%			\import{./papers/paperB/sections/StronConvergenceSSLSM/}{GeneralizedVanDerPol}
	\section{Stochastic Lorenz Equation}
			\import{./papers/paperB/sections/StronConvergenceSSLSM/}{StochasticLorenz}
%	\section{Numerical Methods for Second Order Stochastic Differential Equations}
%		\import{./papers/paperB/sections/StronConvergenceSSLSM/}{SecondOrderLinearAnalysis}
%	\section{Linear Oscillator}
%		This model was studied in \cite{Melbo2004}.
		Melbø Aslaug, H. Strømmen, Higham Desmond. 
		Numerical simulation of a linear stochastic oscillator with additive noise
			\begin{align}
				dx(t) &= y(t)dt\\
				dy(t) &= -x(t)+h dW(t) \qquad h>0.
			\end{align}
		The Steklov Scheme is under construction.
%	\todo{Put here the stochastic extension of the Bone Remodeling model.}
	\pagebreak
	\section*{\refname}
	\bibliographystyle{plainnat}
	\bibliography{bib/PhdThesisBib}
\appendix
\end{document}
