\section{Introduction}
	In the last decades, stochastic differential modeling has become a rapidly-growing research area. Historically, it 
	appeared as an extension of the deterministic differential modeling of over-idealized situations with fluctuating 
	behavior of the analyzed physical phenomenon.  Actually, it is an important research area by itself that describes 
	important phenomena such as turbulent diffusion, spread of diseases, genetic regulation, motion of 
	particles etc. \cite{Allen2007, Gardiner2009, VanKampen1992}. 
		We can obtain the explicit solution of only few stochastic differential equations (SDEs),	therefore developing accurate stochastic
	numerical approximations represent  an option to analyze and confirm (by simulation) the nature of a 
	stochastic model. Stochastic numerics allow the  analysis of some  model properties that are
	difficult or impossible to measure experimentally in laboratories, for example its
	long-time behavior. In this cases, we require that a numerical solvers
	be able to reproduce asymptotic behavior like mean square stability \cite{Higham2000,Higham2000b,Saito1996a},
	usually, a linear  analysis can be considered as the first step for understanding a method, but it is not 
	an indicator of qualitative behavior on the nonlinear case \cite{kloeden1999towards}. Thus, some theoretical 
	work on asymptotic	stability has appeared for nonlinear SDEs \citet{Bokor2003} \citet{Buckwar2011a}.  
	
		The first methods for solving SDEs were stochastic extensions of deterministic algorithms,
	for example schemes as the Euler-Maruyama (EM), Taylor and Runge-Kutta \cite{Bokor2003,Burrage2004, Kloeden1992}.
	Unfortunately, sometimes their asymptotic stability conditions are very restrictive, considers for
	example the Brownian Dynamics Simulations, here the Euler-Maruyama discretization is the standard method to 
	solve the Langevin equation that describe the motion of particles \cite{Braanka1998, Bussi2007, Ermak1978}. 
	However, the operation time step size of this scheme has to be pint-size, otherwise the scheme becomes unstable. 
	Now, the construction of methods focus on structural or dynamic properties of a specific SDE.
	Some examples are the balanced methods for stiff SDE \cite{Milstein1998a} the quasi-symplectic schemes for
	stochastic Hamiltonian systems \cite{Milstein2003} and SDEs with small noise \cite{Buckwar2006b}.
	
		Numerical convergence and stability are well understood for SDE with globally Lipschitz 
	continuous coefficients, which discard  many important models from applications.
	Moreover, \citeauthor*{Hutzenthaler2009} report in \cite{Hutzenthaler2009} that if a SDE
	has drift or diffusion,	which grows faster that a linear function, then the EM diverges in strong and weak sense.
	This result opens a new chapter on the design of numerical methods ---
	stochastic models in applications as Finance, Biology and Physics use SDEs with locally Lipschitz 
	coefficients. In addition, \citeauthor*{Giles2008} proposes in \cite{Giles2008} a new variance reducing technique
	that relies in strong numerical convergence, which optimizes the traditional Monte Carlo simulation. Thus, 
	developing	explicit schemes, which converges in strong sense with super-linear coefficients attracts the right 
	now attention.
	
		Recently research has been focused on modifying the EM method to obtain strong convergence  under 
	the previous conditions keeping its simple structure and  its low computational cost. Several methods have been 
	developed in this direction:  the family of  Tamed schemes
	\cite{Hutzenthaler2012a, Wang2011, Zong2014,Hutzenthaler2015}, 
	a special type of balanced method \cite{Tretyakov2013},  the stopped scheme \cite{Liu2013a} .
	For SDE with super-linear diffusion, \citeauthor{Mao2013a} provided results
	for the strong convergence of implicit methods as the Backward-Euler-Maruyama. However, the convergence of explicit 
	schemes for SDEs with super-linear growth is still under development. Works on this subject  are
	\citeauthor{Mao2015} \cite{Mao2015} with the truncated Euler method and \cite{Sabanis2015} with a new kind of
	tamed scheme. There, the strong convergence of the proposed method
	is proved using the theory developed by in \citet*{Higham2002b} or by means of  the new 
	approach given by  \citet{Hutzenthaler2015}.
	Both techniques prove strong convergence by verifying boundedness moments of the numerical and 
	analytical solution of the underlying SDE. In spite of the recent work in this subject,  it is still necessary
	to get more accurate numerical methods for SDE under super-linear growth and 
	non-globally Lipschitz coefficients.

\section{Main Results}
		Our main contribution follows two lines of research. The first one is to design an explicit numerical
	scheme with good stability properties. We focus on explicit methods because we are interested on applications of 
	Brownian Dynamics, so we seek a simple and fast numerical solver. Also we require a stable scheme 
	in order to obtain  simulations for long periods of time. For example in Brownian Dynamics, the self-diffusion
	coefficient is an asymptotic property. We propose the Steklov method, which is a stochastic 
	extension of an exact deterministic numerical scheme. 
		
		The second line consists in generalize the above scheme to a multidimensional setup and with more general 
	coefficients. Thus, we propose the Linear Steklov (LS) scheme. This explicit method is based on  a linear version 
	of the Steklov average with a split-step formulation. We prove for this scheme a one-half 
	convergence order with a one sided Lipschitz condition and polynomial growth on the drift; and a globally 
	Lipschitz condition on the diffusion. Also we provide numerical evidence that this method is suitable for problems 
	with super-linear growth diffusion  where other methods have failed. 
	 
\section{Methodology}
		
	\subsection*{Chapter 2 --- Preliminaries}
			
			After the above introduction, in this chapter  we present an overview of basic results from probability 
		theory that we will need in order in order to set our framework.
		Next we state useful concepts and theorems from stochastic process and stochastic  calculus. 
		Finally, we give some important qualitative properties of numerical methods for stochastic differential 
		equations.
	
	\subsection*{Chapter 3 --- Steklov method for scalar SDEs with Globally Lipschitz coefficients}
	
			The chapter contains our first contribution ---the Steklov method. 
		First we develop a new numerical method with asymptotic stability properties for solving stochastic 
		differential equations (SDEs). The foundations for the new solver are the Steklov mean and an exact 
		discretization for the deterministic version of the SDEs. Second strong consistency and convergence properties 
		are demonstrated for the proposed method. Moreover, a rigorous linear and nonlinear asymptotic stability 
		analysis is carried out for the multiplicative case in a mean-square sense and for the additive case in a 
		path-wise sense using the pullback limit. In order to emphasize the characteristics of the Steklov 
		discretization we use as benchmarks the stochastic logistic equation and the Langevin equation with a nonlinear 
		potential of the Brownian dynamics. We show that the Steklov method has mild stability requirements and allows 
		long-time simulations in several applications.
	
	\subsection*{Chapter 4 --- Steklov Method for SDEs with Non-Globally Lipschitz Continuous Drift}
	
			In this chapter we present an explicit numerical method for solving stochastic differential equations with
		non-globally Lipschitz coeffcients. A linear version of the Steklov average under a split-step formulation 
		supports our new solver. The Linear Steklov method converges strongly with a standard
		one-half order. Also, we present numerical evidence that the explicit Linear Steklov reproduces almost surely 
		stability solutions with high-accuracy for diverse application models even
		for stochastic differential systems with super-linear diffusion coefficients.
		
	
	\subsection*{Chapter 5 --- Conclusions and future work}
	
			Finally, in this Chapter  we restate our contribution followed by conclusions and discuss 
		possible future directions for our research.