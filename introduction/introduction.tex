\section{Introduction}
		Stochastic differential modeling becomes  a  rapidly-growing
	research area. It appears  as extension of deterministic models with over-idealized fluctuations, 
	but, actually, modeling with stochastic differential equations (SDE)  permits to capture 
	uncertainty in	almost all phenomena --- from price in markets until motion of particles \cite{Allen2007, 
	Gardiner2009, VanKampen1992}. 

		However, we can obtain the explicit solution of only few SDEs. So, developing accurate stochastic
	numerical approximations, represents  the first option to study and confirm (by simulation) the nature of a 
	stochastic model. Stochastic numerics allows the  analysis of some properties that are
	difficult or impossible to measure experimentally in laboratories, for example,
	long-time behavior. In this cases, we require that a numerical solvers
	be able to reproduce asymptotic behavior like mean square stability \cite{Higham2000,Higham2000b,Saito1996a},
	usually, a linear  analysis can be considered as the first step for understanding a method, but it is not 
	an indicator of qualitative behavior on the nonlinear case \cite{kloeden1999towards}. Thus, some theoretical 
	work on asymptotic	stability has appeared for nonlinear SDEs \citet{Bokor2003} \citet{Buckwar2011a}.  
	
		The first methods for solving SDEs were stochastic extensions of deterministic algorithms,
	for example schemes as the Eler-Maruyama, Taylor and Runge-Kutta \cite{Bokor2003,Burrage2004, Kloeden1992}.
	Lamentably, sometimes their asymptotic stability conditions are very restrictive, for
	example in Brownian Dynamics Simulations, the EM discretization, is known
	as the Conventional method for Brownian Dynamics (CBD), because is the standard method to 
	solve the Langevin equation that describe the motion of particles	\cite{Braanka1998, Bussi2007, Ermak1978}. 
	However, the operation time step size of this scheme has to be pint-size, otherwise the scheme becomes unstable. 
	Now, the construction of methods focuses on structural or dynamic properties of a specific SDE.
	Some examples are the balanced methods for stiff SDE \cite{Milstein1998a} the quasi-symplectic schemes for
	stochastic Hamiltonian systems \cite{Milstein2003} and SDEs with small noise \cite{Buckwar2006b}.
	
		Even numerical convergence and stability are well understood on SDE with globally Lipschitz 
	continuous coefficients, this setup discard  many important models from applications.
	Moreover, \citeauthor*{Hutzenthaler2009} reports in \cite{Hutzenthaler2009} that if a SDE
	has drift or diffusion,	which grows faster that a linear function, then the EM diverges in strong and weak sense.
	This result opens a new chapter on the design of numerical methods ---
	stochastic models in applications as Finance, Biology and Physics use SDEs with locally Lipschitz 
	coefficients. In addition, \citeauthor*{Giles2008} propose in \cite{Giles2008} a new variance reducing technique
	that relies in strong numerical convergence, which optimize the traditional Monte Carlo simulation. Thus, developing
	explicit schemes, which converges in strong sense under super-linear coefficients attracts the right now attention.
	
		In this line, recently research has been focused on modifying the EM method to obtain strong convergence  under 
	these conditions keeping its simple structure and  its low computational cost. Several methods have been developed 
	in this direction:  the family of  Tamed schemes
	\cite{Hutzenthaler2012a, Wang2011, Zong2014,Hutzenthaler2015}, 
	a special type of balanced method \cite{Tretyakov2013},  the stopped scheme \cite{Liu2013a} .
	For SDE with super-linear diffusion, \citeauthor{Mao2013a} provide results
	for the strong convergence of Implicit methods as the BEM. However, the convergence of explicit schemes for
	SDEs with super-linear growth is still under development, in this line appears the works of
	\citeauthor{Mao2015} \cite{Mao2015} with the truncated Euler method and \cite{Sabanis2015} with a new kind of
	tamed scheme. In these works, the strong convergence of the proposed method
	is proved using the theory developed by in \citet*{Higham2002b} or by means of  the new 
	approach given by  \citet{Hutzenthaler2015}.
	Both techniques prove strong convergence by verifying boundedness moments of the numerical and 
	analytical solution of the underlying SDE. In spite of the recent work in this subject,  it is still necessary
	to get more accurate numerical methods for SDE under super-linear growth and 
	non-globally Lipschitz coefficients.

\section{Main Results}
		Our main contribution follows two principal lines.  The first is to design a explicit numerical
	scheme with good stability properties. We focus on explicit methods because we are interested on applications of 
	Brownian Dynamics, so we want a simple complex and fast numerical solver. Also we want a stable scheme 
	because we need simulation at long-time. For example in Brownian Dynamics, the self-diffusion
	coefficient is an asymptotic property. In this line we propose the Steklov method, which is the stochastic 
	extension of an exact deterministic numerical scheme. 
		
		The second line consist in generalize the above scheme to a multidimensional setup and under more general 
	coefficients. In this direction, we propose the Linear Steklov scheme (LS). This explicit method puts together 
	ideas of implicit split schemes and a linear version of the Steklov average. We prove for this scheme a one-half 
	order of convergence under a one sided Lipschitz condition and polynomial growth on the drift; and a globally 
	Lipschitz condition on the diffusion. Also we provide numerical evidence that this method works of diffusion with 
	super-linear growth, and in a specific example improves the Tamed Euler family of schemes. 
	 
\section{Outline}
		After this introduction we present an overview of results that we will need in order to discuss our 
	results. So, in chapter 3 we deals with the construction of the Steklov method and prove that have competitive
	stability properties.	This new solver put together ideas from exact numerical 
	approximation \cite{Matus2005}, implicit schemes and the EM method. Since in deterministic context we know that
	this schemes solves exactly the linear problem 
	$
		\frac{dx}{dt} = \lambda x,
	$
	we expect to produce a scheme with acceptable linear stability, in fact, we confirm this for the scalar case
	and prove its strong convergence under classic globally Lipschitz conditions. 
	In chapter 4 we propose an extension of the Steklov method in two directions:
	a multidimensional setting and more general coefficient. Here we prove convergence rates and provide a 
	numerical evidence that the scheme works for super-linear growth diffusions. 
	Finally in Chapter 5 we discuss a possible future direction for our research.